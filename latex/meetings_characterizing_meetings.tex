\subsection{Characterizing Meetings}
% relevance of this section:
Characterizing meetings is critical to distinguishing which norms are applicable, and what people expect from the different formats. 


Types of meetings: internal meetings, customer meetings, conferences, scheduled one-on-ones, impromptu walk-around.   

% https://graphthinking.blogspot.com/2019/12/what-is-purpose-of-this-meeting.html
Many potential purposes of a meeting:
\begin{itemize}
    \item To gather input from attendees
    \item To make a pronouncement to attendees
    \item To educate
    \item To brainstorm ideas
    \item To make progress towards an objective
\end{itemize}
When the purpose is not explicitly stated, confusion arises. 
When multiple purposes occur in one meeting and the transition is not explicitly stated, confusion arises.
The reason for this confusion is that the assumptions and expectations and norms of each purpose are different. When the attendees don't know the purpose or the purpose shifts, the expected behaviors and roles are unclear. 

An attendee can ask what the purpose of the meeting is during the meeting but that is generally considered rude. An attendee can try to deduce the purpose of a meeting, but this takes time and attention and can result in the wrong conclusion. An attendee can try to set the purpose of the meeting during the meeting, but this can conflict with the intent of other attendees. 

A meeting's purpose can shift during a meeting. If done intentionally, the changes should be stated explicitly. Otherwise an attendee may continue to operate under the previous set of expectations rather than the current norms. 



% https://graphthinking.blogspot.com/2014/12/how-to-understand-meetings-at-work.html
Level of formality, start time (early | on time | late), 
end time (early | on time | late), utility, 
duration, number of attendees, number of speakers, number of participants.


What is the purpose of a meeting?

meetings involve people, either known or strangers
meetings involve information, either relevant or irrelevant. Relevant information is either new or related to previous work
meetings either have a leader or no leader (brainstorming). If there's a leader, the leader may be disseminating info to participants, or gathering information from attendees


\subsubsection{Phases of a meeting}

\begin{enumerate}
    \item Establish understanding of what the topic is. Without a shared focus and a common goal a meeting is unlikely to be productive. 
    \item Set aside topics that are not the focus. Either discuss outside the current meeting or defer to a later meeting. Managing scope is critical to accomplishing the goal. 
    \item Establish a shared language specific to the topic and to the participants
    \item Establish each participant's viewpoint on the topic
    \item Brainstorm options
    \item Build consensus or nominate a decider
\end{enumerate}