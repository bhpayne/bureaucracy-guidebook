\subsection{Dilemmas and Trilemmas\label{sec:dilemma_trilemma}}

A simple decision is one that has one good choice and one or more bad choices. Then the goal is to identify which is the good choice and select that option.

A complex decision may have many choices, and there are might not be a best option. Then a Pareto frontier might exist where trade-offs can be made. 

When a decision has two viable options, that presents a \href{https://en.wikipedia.org/wiki/Dilemma}{dilemma}. The name for the case with three viable options is a \href{https://en.wikipedia.org/wiki/Trilemma}{trilemma}.

The following are presented as dilemmas, but that oversimplifies both the continuous nature of the trade-off and the alternative creative approaches to a specific solution. The reason to be aware of these dilemmas to be able to ponder them prior to the pressure of real-time decision making. 

\ \\

\textbf{Number of rules: high versus low}:\\
The more rules that exist the more likely it is that someone will find a way to exploit them to their own advantage.
The fewer rules that exist the more likely it is that someone will try to get away with something bad


\ \\

\textbf{Speed of task completion: fast versus slow}:\\
Do a task quickly (and perhaps ineffectively or inefficiently or even wrong) versus taking time to do the task right (with opportunity cost).

Methodical well-planned design towards getting the right solution versus implementing a solution quickly to address urgent needs


\ \\

\textbf{How much data to gather for a decision: a lot or a little}:\\
Knowing what to do versus the cost of observability. Collecting data is expensive. Lack of data results in decisions based on oversimplified assessment

\ \\

\textbf{Scope of impact: broad versus narrow}:\\
Niche impact means less politics

\ \\

\textbf{staffing flexibility versus efficiency}
Have a small amount of staffing versus cover all edge cases (resilient to changing demands by having slack resources)

bureaucratic inefficiency, or spare capacity that enables resilience to change?
Efficient can be fragile is requirements change


\ \\

\textbf{services that are necessary but not central: in-house versus external dependency}:\\
for non-central services required by the organization, in-house versus outside provider. In either case, the organization is screwed.

\ \\

\textbf{Decision making in a hierarchy: lower versus higher}:\\
the higher up a decision is made, better scope and the decider will have. That decider will have less skin in the game and be less well informed. Pushing decisions down means more inconsistency

\ \\

\textbf{Services within an organization: redundant market versus monopoly}: \\
redundant services to choose from to enable customers to choose the best service (using a market model within the organization) versus efficiency of a single service which might not meet the needs of all customers

solution A exists but doesn't meet my needs. 
rather than tweak A, re-invent solution B which mostly overlaps with A but has independent development and support. 


\ \\

\textbf{choose fast inexpensive good}

\href{https://en.wikipedia.org/wiki/Project_management_triangle}{Project management triangle}