\subsection{Communicating Professional Sense of Vulnerability}

Emotionally vulnerable communication is personal. You are exposing personal issues. This personal emotional vulnerability deepens relationships. The risk is that information could be used to manipulate or harm. 

Professional vulnerability is about being transparent about bureaucratic issues. Processes and incentives are reasonable topics of conversation among professionals. The ramifications are felt by individuals, but a bureaucratic process aimed at an individual would be unprofessional. 

Discussions of internal intrigues of an organization are a form of gossip among professional. \href{https://en.wikipedia.org/wiki/Gossip}{Gossip} can be constructive (finding aspects to remedy) and lead to insights and shapes cultural expectations within the bureaucracy. As with personal gossip, professional gossip can risk harm if used against the organization. 

As with personal vulnerability, professional vulnerability involves learning who to share what information with when. In interacting with a new-to-me person, I experiment by being professionally vulnerable and see whether they reciprocate or at least explore the topic with me. Being open and direct and curious helps the person I'm talking with feel more comfortable. Shared introspection is the objective. 

As an example, I say something like, ``Perhaps the reason behind (observation) is (reason 1) or (reason 2).'' That gives the other person a chance to brainstorm with me without committing to a position. If the other person is unwilling to explore in depth, reverting to safer topics is easy.

Being vulnerable does not mean the other person will reciprocate. That is a risk on your part. 