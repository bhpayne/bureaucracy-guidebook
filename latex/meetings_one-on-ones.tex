\subsection{One-on-one check-in meetings}

% https://graphthinking.blogspot.com/2021/05/the-agenda-for-one-on-one-meeting.html

One-on-one meeting questions for helping the manager understand the team member's status. 
The one-on-one check-in should be tailored to the phase of the employee's progression. 
\begin{enumerate}
    \item \textit{Name of phase}: \underline{Welcome to the team!}\\
    \textit{Scenario: New team member, either new to the team or new to the organization. }\\
    Here the focus of the one-on-one is to ensure a smooth on-boarding process. Get them up-to-speed on the technical challenges, professional norms, and integrated with other team members. Resolve administrative blockers: Does the employee have the necessary computer log-in accounts? Do they have an email account? Are they on the mailing list? \\
    Questions:
    \begin{itemize}
        \item What are the objectives for the team?
        \item What items on the onboarding checklist are not yet completed?
        \item Who have you met on the team? What is your understand of their role on the team?
    \end{itemize}
\textit{The duration of this phase could last between a day and two weeks.}
    \item \textit{Name of phase}: \underline{Initial contributions}\\
    \textit{Scenario: Team member is responsible for small tasks. }\\
    The one-one-one is for discussions on training and planning and task reviews. Characterized by the team member being dependent on others for their success. In this phase the employee collaborates on tasks.\\
    Questions:
    \begin{itemize}
        \item What are the objectives for the team?
        \item What are your task objectives?
        \item What are you expecting to deliver to the team? When? 
        \item What dependencies does that deliverable have (external to the team or internal to the team)?
    \end{itemize}
\textit{The duration of this phase could last a few months to years.}
    \item \textit{Name of phase}: \underline{Experienced contributor}\\
    \textit{Scenario: Team member is responsible for large tasks (which get broken into subtasks). }\\
    The one-on-one is to help the team member define their success. Activities include planning, resource allocation, assessment. Characterized by the need to coordinate with others on the team or other teams. Team member understands task scope and intent and relevant processes. Team member decomposes task into subtasks.\\
    Questions:
    \begin{itemize}
        \item How do the artifacts you're working on support your plan for the team's progress?
        \item What dependencies does that deliverable have (external to the team or internal to the team)?
        \item What insights do you have about the team or organization?
        \item What insights do you have about the relevance of the task relative to the intent of the organization?
        \item What should management be doing to enable the team's success?
    \end{itemize}
\textit{The duration of this phase could could be the rest of a career.}
    \item \textit{Name of phase}: \underline{Facilitator}\\
    \textit{Scenario: Facilitating the productivity of others.}\\
    Rather than being task-oriented, this team member supports coworkers. \\
    Questions:
    \begin{itemize}
        \item What observations from mentoring team members do you have?
        \item What collaborations should we be fostering?
    \end{itemize}
    \item \textit{Name of phase}: \underline{Peer}\\
    \textit{Scenario: Peer check-in.}\\ 
    This one-on-one is a form of mentorship. The value of the exchange is to get a different perspective and to hold each other accountable.
\end{enumerate}

TODO: How does the team member and the supervisor know when the next phase is appropriate?

TODO: What are the thresholds for change?

In addition to phase-specific questions, there are questions appropriate to all phases. A manager should ask the team member
\begin{itemize}
    \item What have you been successful with since we last met?
    \item What is blocking our team's progress?
    \item What are your plans?
    \item How are you collaborating with the rest of the team?
    \item If there was just one thing you could change about our organization, what would it be and why?
    \item How do you plan to train your coworkers on topics you understand and they don't?
    \item What have you learned in the past month?
    \item What are the biggest risks for the team?
    \item What's limiting your productivity?
\end{itemize}
Responding to these questions takes time (an hour) and willingness to be open. 


\ \\

A one-on-one meeting requires preparation. Before the meeting the team member should document
\begin{itemize}
    \item What was discussed previously
    \item What progress has been made since the previous meeting
    \item What is blocking progress
\end{itemize}

\ \\

TODO: https://graphthinking.blogspot.com/2021/09/notes-from-half-day-course-on-coaching.html

TODO: https://graphthinking.blogspot.com/2021/07/thought-terminating-concepts-in.html

\ \\

https://news.ycombinator.com/item?id=30152268

\ \\

https://news.ycombinator.com/item?id=22341138
https://github.com/VGraupera/1on1-questions

