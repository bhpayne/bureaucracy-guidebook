\subsection{Ideas for Innovation within a Bureaucracy\label{sec:innovation}}



0) people observe problems and challenges in their environment
1) share ideas for innovation
2) implementing these suggestions require either 
    - a change of existing processes or 
    - new processes or 
    - an investment of work (which may not succeed -- there's risk)
3) implementation would take you away from your current work, and it might involve skills you don't have
4) someone else implements it, so they get the credit for having done the work
5) if the idea saves money or time, there's no monetary reward

Your idea could fail because it's not a good idea. It could also fail because someone doesn't like you, or the idea doesn't account for some dependency you weren't aware of, or it might conflict with other changes in progress.

The solution is you need to build a coalition of people willing to fight for the idea on your behalf so that when you're not present, the idea is still proceeding to implementation. That puts the threshold at ``so important other people are willing to pause whatever they were working on and take up your cause.''


Which leads to external observers to conclude something like the following simplification:
\begin{quote}
Bureaucracy destroys initiative. There is little that bureaucrats hate more than innovation, especially innovation that produces better results than the old routines. Improvements always make those at the top of the heap look inept. Who enjoys appearing inept?\footnote{Frank Herbert (1987). ``Heretics of Dune'', page 201, Penguin}
% https://www.azquotes.com/quote/453163
\end{quote}
