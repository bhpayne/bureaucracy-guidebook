\section{Ideas for Innovation within a Bureaucracy\label{sec:innovation}}

As a bureaucrat in an organization, you will observe deficiencies and challenges. You could ignore the issues, complain but not take action, work around (avoid) the issues, or try to improve your organization. This section focuses on the last option.

Sometimes the issue and the apparent fix seem straightforward. If the situation has persisted for a long time, consider why the conditions have remained as is. 
Instead of framing your idea with, ``Isn't it obvious that things would be better if ..." learn why what seems obvious doesn't happen. 
Learn the history. Talk to participants. If there is agreement that the situation is harmful, why has no one else tackled it? If previous attempts to innovate were tried and failed, why did they fail? What is novel about your innovation?

If you understand the conditions creating a challenge, and the history of why other approaches failed, and you have an innovative response, consider the stakeholders. 
Innovation is necessarily disruptive to incumbent processes and teams and social power. 
Do the incumbents benefit from the proposed innovation? Have they said they will benefit, or are you projecting their response? 
Innovation that benefits the organization but harms incumbents is less likely to succeed. 

In addition to the social hurdles, innovation requires taking risks (otherwise the effort is merely incremental improvement). Risk in an organization means losing time or resources, potential harm to reputation, legal consequences. 

Once you decide to innovate, and after you've learned the history and talked to participants, you have choices about how to innovate:
\begin{itemize}
\item Innovate while ignoring the bureaucracy. Your scope is limited to what you can accomplish alone. Integration is tough and causes frustration.
\item Fight to create a space free from bureaucracy to enable innovation. You've intentionally isolated innovators from resources present in the organization, and integration back into the org is now more challenging. 
\item Leverage existing hierarchy and processes. Can be frustrating to work with bureaucrats who are not as focused as you are.
\end{itemize}


A recap of the innovation lifecycle in a bureaucracy:
\begin{enumerate}
    \item You observe problems and challenges in your environment. This manifests as complaints from both the people directly harmed and observers who see inefficiency.
    \item You share ideas for innovation and gets feedback. Build a coalition of people willing to fight for the idea on your behalf
    \marginpar{[Tag] Actionable Advice}
    so that when you're not present, the idea is still proceeding towards implementation.  That puts the threshold at ``so important other people are willing to pause whatever they were working on and take up your cause.''
    \item Implementing these suggestions require either a change to existing processes or new processes or an investment of work. These changes may not succeed -- there's risk. Your idea could fail because it's not a good idea. It could also fail because someone doesn't like you, or the idea doesn't account for some dependency you weren't aware of, or it might conflict with other changes in progress.
    \item If you do decide to invest effort, the activity takes you away from your current work. Implementing the change might involve skills you don't have; learning those skills takes time. Carrying out the activity with new skills increases the likelihood of novice mistakes.
    \item If someone else implements the idea they get the credit for having done the work.
    \item If the idea saves money or time, there's typically no monetary reward. Recognition and benefit to your reputation isn't required as part of the change process. 
\end{enumerate}

There are many barriers in that lifecycle. The problem has to be observable to someone willing to invest effort in change. That person has to build a coalition of stakeholders. If the person isn't negatively harmed, that person may also lack clear benefit from resolving inefficiency. 

\ \\

In addition to the work of implementing the change, there is an administrative overhead of documenting the reason for the change. 
Subjective decisions mean choices have to be defensible. 
The need for defensible justifications result in conservative decisions and risk aversion and a decrease of motive for innovation. 

\ \\

Don't bother buying new technology because there's a lot of risk and we already know how to use the existing technology. (Same applies to innovation with policies.)

The people in decision making positions in the hierarchy have more experience with existing technology and are therefore biased against novel technology. Also, getting burned a few times on innovation makes people more conservative.

With limited time and staff available within the organization, experienced decision makers are typically not on the bleeding edge of the \href{https://en.wikipedia.org/wiki/Gartner_hype_cycle}{hype cycle}.


These barriers lead to external observers to conclude something like the following simplification:
\begin{quote}
Bureaucracy destroys initiative. There is little that bureaucrats hate more than innovation, especially innovation that produces better results than the old routines.
\marginpar{[Tag] Folk wisdom}
Improvements always make those at the top of the heap look inept. Who enjoys appearing inept?\footnote{Frank Herbert's 1987 ``Heretics of Dune'', page 201}%, Penguin}
% https://www.azquotes.com/quote/453163
\end{quote}

