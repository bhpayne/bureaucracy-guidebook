% see https://en.wikibooks.org/wiki/LaTeX/Glossary
% and https://www.overleaf.com/learn/latex/Glossaries

\newglossaryentry{organization}{
  name={organization},
  plural={organizations},
  description={an assembly of teams. The name of the concept might be a corporation, an agency, a department, a bureau, or any other aggregation of smaller organizations}
}

\newglossaryentry{culture}{
name={culture},
plural={cultures},
description={norms, expectations around interaction among people}
}

\newglossaryentry{stakeholder}{
name={stakeholder},
plural={stakeholders},
description={a person who cares about the process or the outcome; distinct from a participant. Subcategories: Who is impacted?
Who has control?}
%descriptionplural={people who cares about the process or the outcome; distinct from participants}
}

\newglossaryentry{participant}{
name={participant},
description={a person who is expected to take action or make contribution}
}

\newglossaryentry{essential bureaucracy}{
name={essential bureaucracy},
text={Essential bureaucracy},
description={the minimum processes and staffing and skills necessary to address the complexity of the problem space}
}

\newglossaryentry{bureaucratic debt}{
name={bureaucratic debt},
plural={bureaucratic debts},
text={Bureaucratic debt},
% https://graphthinking.blogspot.com/2017/02/schedules-slip.html
description={the cost of work need to change a process caused by choosing an easy solution now instead of using a better approach that would take longer. Bureaucratic debt is a trade-off, conscious or unconscious, of what work to do and risks to take. More effort (work, time) could be spent building a better product, but customers want solutions now.}
}

\newglossaryentry{subject}{
    name={subject},
    plural={subjects},
    description={the person experiencing bureaucracy. See \cite{1983_Lipsky} page XIV}
}

\newglossaryentry{Prisoner's dilemma}{
    name={Prisoner's dilemma},
    description={Two or more people with incomplete information of a situation will make suboptimal choices compared to someone with perfect knowledge of the situation}
}

\newglossaryentry{thought terminating}{
name={thought terminating},
description={\href{https://en.wikipedia.org/wiki/Thought-terminating_clich\%C3\%A9}{thought terminating statements} initially sound reasonable but, upon reflection and analysis, are incorrect}
}

\newglossaryentry{presence creates priority}{
name={presence creates priority},
description={being physically at a person's desk motivates that person to respond better than calling them or emailing them}
}

% https://graphthinking.blogspot.com/2021/07/bureaucracy-book-outline.html
\newglossaryentry{bureaucrat}{
    name={bureaucrat},
    plural={bureaucrats},
    description={the person who is a member of an organization and is responsible for subjective implementation of policy for the organization. Conventional examples of a bureaucrat role: teacher, police, government employee}
}

\newglossaryentry{simple decision}{
  name={simple decision},
  description={has one correct or beneficial choice and one or more wrong or harmful choices.}
}

% https://tex.stackexchange.com/questions/69567/uppercase-word-in-glossary-lowercase-in-text
\newglossaryentry{bureaucracy}{
    name={bureaucracy},
    plural={bureaucracies},
    text={bureaucracy},
    description={
    An organization of bureaucrats comprises a bureaucracy. A bureaucracy facilitates coordination of stakeholders. 
    Bureaucracy is how large organizations make distributed decisions using distributed knowledge.    \\
    Everything in a bureaucracy is made up by other participants. \\
    Bureaucracy is a macroscopic phenomenon emergent at sufficient scale. The scale is important because there is no longer dependence on individual relationships. \\
    Bureaucracy arises when there is no common objectively quantifiable feedback mechanism for individual participants in the organization.\\
    Bureaucracy is a wicked problem}
}

\newglossaryentry{visible bureaucracy}{
name={visible bureaucracy},
    %name={bureaucracy, visible},
    description={Processes are written down and can be discovered by stakeholders}
}
\newglossaryentry{invisible bureaucracy}{
name={invisible bureaucracy},
    %name={bureaucracy, invisible},
    description={Processes are known to some stakeholders and are conveyed verbally to some of the other stakeholders}
}

\newglossaryentry{process}{
name={process},
plural={processes},
% aka procedure
% https://graphthinking.blogspot.com/2017/02/breaking-down-bureaucracy-process-roles.html
description={A task broken into a specified set of subtask dependencies, typically with subtasks in a sequence. 
Each task is associated with the application of policies enforced by bureaucrats. 
% Also known as a procedure. 
Two distinguishing features in the context of bureaucracy are authorization and justification.  
A process has inputs and outputs. 
A process can be decomposed into other processes. 
Processes operate on both information and tangible objects. 
Processes require \href{https://en.wikipedia.org/wiki/Work_(physics)}{work} and time. 
Processes are carried out by people or machines.}
}

