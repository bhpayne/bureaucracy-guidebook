\section{Fundamentals of Bureaucracy}

\subsection{What is bureaucracy?}
While you may know it when you see it or experience it, for this book a definition is useful. 
\gls{bureaucracy} is coordination of stakeholders. This concept is most visible for complex, long lasting, and recurring situations involving many people. The apparent friction can be lower when there are only a few people involved ("I'm just talking to my collaborator"), but there is a continuous gradient. 

\subsection{What does Bureaucracy imply for you?}
Bureaucracy is neither good nor bad. Bureaucracy is not tied to politics, or any specific institution (corporations, governments, academics). Bureaucracy is not defined to be efficient nor, does it have to be inefficient. Bureaucracy is not restricted to paperwork, or record keeping, or quantification, or gathering metrics. 

Bureaucracy is about delegation of control, communication, decision making, coordination, and processes. In that context, wouldn't it be useful to be skilled at bureaucracy? 

\subsection{Why does bureaucracy exist? Can't we just do the work?}

The minimal scenario to start from is to imagine a single person working on a single task that does not last long (a few minutes), is relatively easy (cognitively and physically and emotionally), and does not recur. Most of what you do occurs outside those limits and thus incurs some concept of \gls{process} (breaking a task into subtasks). Staying with the one-person constraint, a complex task can benefit from being broken into subtasks where order of the subtasks matters. 

% https://graphthinking.blogspot.com/2021/09/why-is-everything-so-hard-in-large.html

What if we completely avoided bureaucracy? Try replacing "bureaucracy" in that question with "coordination of stakeholders". If you avoid coordination of stakeholders, what you get is random interactions. 

What if we minimized bureaucracy? Again, try replacing "bureaucracy" in that question with "coordination of stakeholders". The goal of "minimizing coordination" probably isn't the real objective. To be more precise, a specific objective might be "minimize time spent executing the task" (which takes a lot of coordination prior to the task execution) or "minimize the level of distraction to stakeholders" (chunk the coordination time). Another strategy for minimizing bureaucracy is to reduce the number of stakeholders involved. For a given task complexity, this means having smarter people who have more skills. 


\subsection{Hierarchy of roles}

An \gls{organization} often (though not always) has a defined set of roles, and those roles have different amount of decision authority. 

\subsection{Org chart as a guide and a lie}

org chart identifies roles and the relations among roles. 

lie in the sense that undocumented relationships matter more than the roles

lie in the sense of orientation; see \ref{org-chart-orientation}

\subsection{Approval process}

\subsection{Meetings for coordination}

coordination and signaling

\subsection{Written communication}

Reports, memos, emails are artifacts of bureaucracy. They create evidence and can be used for good or bad. 

