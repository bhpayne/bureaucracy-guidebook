\section{Fundamentals of Bureaucracy}

\subsection{What is bureaucracy?}
While you may know it when you see it or experience it, for this book a definition is useful. 
\gls{bureaucracy} is coordination of stakeholders. This concept is most visible for complex, long lasting, and recurring situations involving many people. The apparent friction can be lower when there are only a few people involved ("I'm just talking to my collaborator"), but there is a continuous gradient. 


\subsection{Why does bureaucracy exist? Can't we just do the work?}

The minimal scenario to start from is to imagine a single person working on a single task that does not last long (a few minutes), is relatively easy (cognitively and physically and emotionally), and does not recur. Most of what you do occurs outside those limits and thus incurs some concept of \gls{process} (breaking a task into subtasks). Staying with the one-person constraint, a complex task can benefit from being broken into subtasks where order of the subtasks matters. 

% https://graphthinking.blogspot.com/2021/09/why-is-everything-so-hard-in-large.html



\subsection{Hierarchy of roles}

\subsection{Org chart as a guide and a lie}

lie in the sense that undocumented relationships matter more than the roles

lie in the sense of orientation; see \ref{org-chart-orientation}

\subsection{Approval process}

\subsection{Meetings for coordination}

coordination and signaling

\subsection{Written communication}

Reports, memos, emails are artifacts of bureaucracy. They create evidence and can be used for good or bad. 

