\subsection{Learning from Failure}

Making progress in a bureaucracy is not a linear sequence of steps. Ideally there is an on-going cycle of trying something, not succeeding, and then applying what you learn towards the next attempt at trying something. This idealized ``fail fast'' cycle does not occur naturally -- the participants have to either have intrinsic motivation or external incentives to make progress. 

The recognition of failure depends on a clear measurement. Do the participants know the measure, and are they able to make the measurement? Having a defined measure of failure and regularly making the measurement relies on having an understanding of the expectations for the situation. Expectations are assumptions about the future.

Assumptions about the future can be categorized as
\begin{itemize}
    \item opinions. For example, interpretation of policy. Policy interpretation can be bent or exceptions can be made. 
    \item guesses; more formally interpolation -- given multiple options, this seems most likely.
    \item experience; more formally extrapolation -- last time this happend, so next time ...
\end{itemize}

Failure with respect to expectations imply thinking about the future so that you can measure your progress (or failure). That is a subset of planning. You can fail if you don't plan, and you can fail if you do plan. This does not indicate that planning is irrelevant. Plans can be overly detailed and prescriptive, or plans can be inadequately specified; both are unhelpful.

If your measurements indicate failure, this could be due to an invalid assumption (with a valid objective) or you may have selected a bad objective. Declaring failure means re-evaluating assumptions, resetting objectives, or giving up on the concept and doing something else.

