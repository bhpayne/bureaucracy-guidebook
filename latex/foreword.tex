% Who this book is for
I wrote this book for me. I knew I didn't know much when I first started my job in a large organization. I learned from my mistakes by reflecting on my (in)actions and consequences. That has been an expensive education; mistakes delay progress and damage relationships.

% from https://graphthinking.blogspot.com/2021/07/bureaucracy-book-outline.html
This book is for you if you are curious about bureaucracies, or you are thinking about working as a bureaucrat, or you are employed as a bureaucrat, or your job is shifting to be more bureaucratic. 


% What you should expect reading this book: 
This is not a book on leadership, managing a team, being a team member, planning, time management, project management, advancing your career, or self-improvement. 

% What is the benefit of reading this book?
Better able to recognize and navigate complex environments, both in your career and outside of work. This perspective can benefit you whether by promotion of title or increase in pay; successful completion of a project; decreased stress of understanding how the world works.

% Caveats
My experiences cannot be generalized to every situation. Some of the observations here may be analogous to your context if you squint hard. 

Nothing in this book is domain specific, nothing is tied to engineering of products, and nothing is applicable solely in science research. While this material is intended to be timeless and generic, it is culturally specific to the USA. As a privileged white male, I did not encounter systemic hurdles in my career. 

% Source of this content: 
This material is based on personal experience, reading published materials, and anecdotes from other people.

% How the book should be read: 
Reading this book front-to-back is feasible. Each section is intended to be stand-alone. The book is intended to spark contemplation. 

