
% Who this book is for

If you don't think of yourself as a bureaucrat, I hope to change your mind on this essential topic. 
 If the idea of bureaucracy has negative connotations for you, I want to convince you that bureaucracy is useful and that you can learn to skillfully navigate bureaucracy.


% from https://graphthinking.blogspot.com/2021/07/bureaucracy-book-outline.html
This book is for you if you are curious about the complex world we live in, or you are thinking about how to productively contribute to society, or you want impactful employment, or your job is not what you expected. 


% What you should expect reading this book: 
Everyone in modern society is a participant in bureaucracy. The purpose of this book is to decrease the surprise of that experience and better arm you emotionally and intellectually for the toil of being a bureaucrat. With focused reflection and a good guidebook, you can improve your skills as a bureaucrat. 

This book does not focus on leadership, managing a team, being a team member, planning, time management, project management, advancing your career, or self-improvement. However, in the process of being a better bureaucrat some lessons may apply in those domains.

% What is the benefit of reading this book?
As a result of reading this book, you will be better able to recognize and navigate complex professional environments, both within your career and outside of work. The perspectives offered in this book can benefit you directly, whether by promotion of title, increase in pay, successful completion of a project, or through decreased stress of understanding how the world operates. Being a more effective bureaucrat can also positive impact the causes you care about and the people you engage with.

If you do not recognize that you are bureaucrat, you're less likely to be successful interacting with those around you. The self-recognition of being a bureaucrat matters; how you behave and what you think your responsibilities are depend on how you label yourself.

Even people who are smart (e.g., they know history, they have memorized capital cities, they can do math) can struggle in the face of complex large-scale systems. Thinking about complex large-scale systems is not part of the education curriculum. This book will help you learn about bureaucracy and lead to an increase of your ability to identify patterns and apply relevant techniques.

% there's no avoiding the issue
Everyone is a bureaucrat because there are no alternatives to bureaucracy for a society. Gaining skills in navigating bureaucracy are helpful both for your own happiness and the well-being of a functioning society. 

Hoping that modern technology will eliminate or reduce bureaucracy is not helpful. Automation and computers merely obfuscate processes and make negotiation more challenging. 

Simplifying interactions with other people to ``this is characterized merely as human relations" is an easier perspective compared to considering bureaucracy and complex systems. 
% this is also stated on page 24 of \cite{1991_Wilson}
However, a simplified view either misses emergent phenomena or mischaracterizes the situation. In either case, your effectiveness is harmed.



% my experience
% I wrote this book for a younger version of me.
 When I first started my job in a large organization I recognized differences between the expectations of the education system I had left and the challenges of a professional environment. Over the years I learned from my mistakes by reflecting on my (in)actions and the consequences. This approach has been an expensive education. My mistakes delayed progress and damaged relationships. The motive in this book is to provide generalizations from my experiences which might benefit the reader.


% Caveats

In my reflections and attempting to draw lessons there is a risk of overanalysis. Sometimes a situation is merely happenstance, and sometimes attempting to extract lessons from randomness is folly. Avoiding conjecture about conspiracy and malice is a fuzzy boundary when insufficient information is available. 

My experiences cannot be generalized to every situation. Some of the observations here may be analogous to your context if you squint. 

Nothing in this book is domain specific, nothing is tied to engineering of products, and nothing is applicable solely in science research or policy development. While this material is intended to be timeless and generic, it is culturally specific to the United States of America in the early twenty first century. There are cultural blindspots not addressed in this book because I did not encounter systemic hurdles in my career as financially privileged white male. 

% Source of this content: 
This material is based on personal experience, reading published materials, and anecdotes from other people. No surveys were taken to support the claims made. No double blind experiments were conducted. 

% How the book should be read: 
Reading this book front-to-back is the default option. With the exception of \S~\ref{sec:define_bureaucracy}, Chapter~\ref{b_throughout_life} provides context for the lifelong experience of bureaucracy. 
\S~\ref{fundamentals_of_b} describes the essentials of bureaucracy. Each section in Chapter~\ref{b_made_of_humans} is intended to be able to be read stand-alone. The content is intended to spark contemplation. 

\ \\

% as per https://tex.stackexchange.com/q/393238/235813
\begin{flushright}
Ben Payne\\
\today\\
United States of America,\\
Earth
\end{flushright}


