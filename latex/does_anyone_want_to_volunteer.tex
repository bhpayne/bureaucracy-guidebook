\section{Does Anyone Want to Volunteer?}

% https://graphthinking.blogspot.com/2020/06/what-to-ask-instead-of-does-anyone-want.html

The facilitator of a meeting asking, ``Does anyone want to volunteer for this task?" to a group will most often be met by silence. 
The lack of engagement can be addressed by not asking for volunteers. The outcome can be higher engagement with the following tactics:
\begin{enumerate}
    \item Ask each person whether they are attending to observe or to volunteer.
    \item Ask each person how much time they are willing to volunteer. Responses could be 0, or 1 hour (non-recurring), or 1 hour per month, or something else.
    \item Ask each person what their goals in the interaction are. It is usually generic (``I want the group to succeed.") but it can be narrow, in which case that gives you something to focus on.
    \item Ask each person what they are good at and what skills they have. Are they good with personal interaction? Writing? Computers? Coordinating? Logistics? Fund raising? Making phone calls?
    \item From the inventory of tasks, are there any that fit both the skills and time? Can the task be scoped to fit the time? If there are multiple candidate tasks for a volunteer, let them pick the task. (If there is no existing task that aligns with their skills, do not create work to be assigned.)
    \item If there are other volunteer working the same tasks, put them in contact with the other participants
    \item Give the volunteer a deadline for the task. The deadline should not be arbitrary -- the deadline should be based on the dependent follow-on tasks. 
    \item Confirm that the volunteer is willing to commit the time to complete the tasks by the deadline
    \item Schedule a check-in with the volunteer before the task deadline to review progress
    \item To create accountability among multiple volunteers, hold a group review to explain how each participant's work contributes to the goals and how work done by one person enables the next task done by someone else. (Enumerate the dependency graph; include deadlines.)
\end{enumerate}
