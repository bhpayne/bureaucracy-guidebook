\section{Bureaucratic Fallacies\label{sec:fallacies}}

There are perspectives that are \gls{thought terminating}. Identifying these enables you to understand both why they are attractive and how they are incomplete.

See also unavoidable hazards in \S~\ref{sec:unavoidable_hazards}.

\ \\

\textit{Bureaucratic fallacy}: \textbf{Bureaucracy is bad}. \\
\textit{Why this feels true}: when a person subjected to bureaucracy has a negative experience, the easiest attribution is to the least-understood aspect -- the bureaucracy.\\
\textit{What this is missing}: Bureaucracy itself is neither good nor bad. 

\ \\

\textit{Bureaucratic fallacy}: \textbf{product-focused narrative}. \\
\textit{Example}: \href{https://www.youtube.com/watch?v=OgVKvqTItto}{School House Rock's I'm Just A Bill}\\
\textit{Why this feels true}: \\
\textit{What this is missing}: 

\ \\

\textit{Bureaucratic fallacy}: \textbf{subject-focused narrative}. \\
\textit{Why this feels true}: \\
\textit{What this is missing}: 

\ \\

\textit{Bureaucratic fallacy}: \textbf{system as an entity}. \\
See \cite{2002_Gall}
\textit{Why this feels true}: \\
\textit{What this is missing}: The problem with treating an org as entity is that apparent behavior is counter-intuitive. But when you break it into individual people their motives make more sense. This merely improve the post hoc rationalization. 

\ \\

\textit{Bureaucratic Fallacy}: \textbf{duration of a task is how long it would take one person to accomplish}. 
\textit{Why this feels true}: \\
\textit{What this is missing}: Fails to account for the overhead of interaction and delays due to asynchronous engagement.

\ \\

\textit{Bureaucratic Fallacy}: \textbf{people learn from their mistakes}. \\
\textit{Why this feels true}: \\
\textit{What this is missing}: Require a low latency feedback loop and incentive to change

\ \\

\textit{Bureaucratic Fallacy}: \textbf{processes are serial}.\\
\textit{Why this feels true}: \\
\textit{What this is missing}: 


\ \\

\textit{Bureaucratic Fallacy}: \textbf{Hard work creates results}.\\
\textit{Why this feels true}: \\
\textit{What this is missing}: 


\ \\

\textit{Bureaucratic Fallacy}: \textbf{Motivations for bureaucrats are categorized as individualistic, tribal, organizational, societal, or humanity}.\\
\textit{Why this feels true}: \\
\textit{What this is missing}: 


\ \\

\textit{Bureaucratic Fallacy}: \textbf{you cannot pay a little and get a lot}; see \href{https://en.wikipedia.org/wiki/Common_law_of_business_balance}{Common law of business balance}. \\
\textit{Why this feels true}: \\
\textit{What this is missing}: This doesn't allow for creative solutions and ignores \href{https://en.wikipedia.org/wiki/Nudge_theory}{nudge theory} from behavioral economics. 