\section{Bureaucratic Fallacies\label{sec:fallacies}}

There are thought patterns and perspectives that are \gls{thought terminating}. 

\ \\

\textit{Bureaucratic fallacy}: \textbf{product-focused narrative}. Example: \href{https://www.youtube.com/watch?v=OgVKvqTItto}{School House Rock's I'm Just A Bill}

\ \\

\textit{Bureaucratic fallacy}: \textbf{subject-focused narrative}. Example: 
\ \\

\textit{Bureaucratic fallacy}: \textbf{system as an entity}. \cite{2002_Gall}
The problem with treating an org as entity is that apparent behavior is counter-intuitive. But when you break it into individual people their motives make more sense. This merely improve the post hoc rationalization. 
\ \\

\textit{Bureaucratic Fallacy}: \textbf{duration of a task is how long it would take one person to accomplish}. Fails to account for the overhead of interaction and delays due to asynchronous engagement.

\ \\

\textit{Bureaucratic Fallacy}: \textbf{people learn from their mistakes}. Require a low latency feedback loop and incentive to change

\textit{Bureaucratic Fallacy}: \textbf{processes are serial}


\textit{Bureaucratic Fallacy}: \textbf{Hard work creates results}

\textit{Bureaucratic Fallacy}: \textbf{Motivations for bureaucrats are categorized as individualistic, tribal, organizational, societal, or humanity}


\textit{Bureaucratic Fallacy}: \textbf{you cannot pay a little and get a lot}; see \href{https://en.wikipedia.org/wiki/Common_law_of_business_balance}{Common law of business balance}. 
This doesn't allow for creative solutions and ignores \href{https://en.wikipedia.org/wiki/Nudge_theory}{Nudge theory} from behavioral economics. 