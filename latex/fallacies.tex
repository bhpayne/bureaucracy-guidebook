\section{Bureaucratic Fallacies\label{sec:fallacies}}

There are perspectives that are \gls{thought terminating}. Identifying these enables you to understand both why they are attractive and how they are incomplete.

See also unavoidable hazards in \S~\ref{sec:unavoidable_hazards}.

\ \\

\textit{Bureaucratic fallacy}: \textbf{Bureaucracy is bad}. \\
\textit{Why this feels true}: when a person subjected to bureaucracy has a negative experience, the easiest attribution is to the least-understood aspect -- the bureaucracy.\\
\textit{What this is missing}: \Gls{bureaucracy} is neither good nor bad. 

\ \\

\textit{Bureaucratic fallacy}: \textbf{There is no point in planning ahead since everything (staffing, funding, purpose, scope) is always changing.}\\
\textit{Why this feels true}: Change can feel disorienting, especially when it is unexpected. \\
\textit{What this is missing}: Preparing for change and thinking ahead about contingencies enables effective use of resources. Have a vision and work towards it while accounting for change. 


\ \\

\textit{Bureaucratic fallacy}: \textbf{Bureaucracy is an aberration, a mistake, due to poor planning or incompetent participants}. \\
\textit{Why this feels true}: TODO\\
\textit{What this is missing}: TODO

\ \\

\textit{Bureaucratic fallacy}: \textbf{Bureaucracy is inefficient}. \\
\textit{Why this feels true}: Expressed by both subjects and bureaucrats who observe seemingly wasteful processes.\\
\textit{What this is missing}: If bureaucracy were truly inefficient (not allocating resources in the most efficient way), then it would be replaced by a more efficient approach. The key is to ask, ``efficient with respect to what metric?'' The metric of money, time, number of people, stability, robustness to perturbation.  Second, what would motivate improved efficiency? Without incentives, change is less likely. 

\ \\

\textit{Bureaucratic fallacy}: \textbf{Bureaucracy is due to malfeasance.}\\
The specific number of malicious bureaucrats ranges from ``all of the participants'' to ``just enough to be problematic.'' \\
\textit{Why this feels true}: There are bad actors. 
\textit{What this is missing}: Bureaucracy is unavoidable, and most of the participants are earnestly trying to help or are not making a positive contribution. Processes within bureaucracy are used to deal with malicious bureaucrats, like isolation or promotion. 

\ \\

\textit{Bureaucratic fallacy}: \textbf{Bureaucracy is a sign of decay from within the org.} \\
\textit{Why this feels true}: TODO\\
\textit{What this is missing}: Bureaucracy is unavoidable emergence in any/every organization.

\ \\

\textit{Bureaucratic Fallacy}: \textbf{If this request can't be expedited, it must not be important}.  \\
\textit{Why this feels true}: Other people would demonstrate they care about what I am working on by prioritizing things I am dependent on.
\textit{What this is missing}: When everything gets prioritized, that's the same as nothing getting priority.

\ \\

\textit{Bureaucratic Fallacy}: \textbf{Duration of a task is how long it would take one person to accomplish}.  \\
\textit{Why this feels true}: TODO\\
\textit{What this is missing}: Fails to account for the overhead of interaction and delays due to asynchronous engagement.


\ \\

% https://graphthinking.blogspot.com/2019/08/two-misleading-simplifications-when.html
\textit{Bureaucratic Fallacy}: \textbf{Consider the average or majority (to the exclusion of outliers)}. \\
Why this simplification is misleading: For sufficiently large ensembles, the outliers alter the outcome. The larger the ensemble, the more significant the role of the outlier minority.

\ \\

\textit{Bureaucratic Fallacy}: \textbf{people learn from their mistakes}. \\
\textit{Why this feels true}: TODO\\
\textit{What this is missing}: Requires a low latency feedback loop and incentive to change.

\ \\

\textit{Bureaucratic Fallacy}: \textbf{processes are serial}.\\
\textit{Why this feels true}: TODO \\
\textit{What this is missing}: TODO


\ \\

\textit{Bureaucratic Fallacy}: \textbf{Hard work creates results}.\\
\textit{Why this feels true}: TODO\\
\textit{What this is missing}: TODO


\ \\

\textit{Bureaucratic Fallacy}: \textbf{Motivations for bureaucrats are categorized as individualistic, tribal, organizational, societal, or humanity}.\\
\textit{Why this feels true}: TODO\\
\textit{What this is missing}: TODO


\ \\

\textit{Bureaucratic Fallacy}: 
\textbf{you cannot pay a little and get a lot}; see \href{https://en.wikipedia.org/wiki/Common_law_of_business_balance}{Common law of business balance}. \\
\textit{Why this feels true}: TODO\\
\textit{What this is missing}: This doesn't allow for creative solutions and ignores \href{https://en.wikipedia.org/wiki/Nudge_theory}{nudge theory} from behavioral economics. 

