\section{Bureaucratic Fallacies\label{sec:fallacies}}

There are perspectives that are \gls{thought terminating}. Identifying these enables you to understand both why they are attractive and how they are incomplete.

See also unavoidable hazards in \S~\ref{sec:unavoidable_hazards}.

\ \\

\textit{Bureaucratic fallacy}: \textbf{Bureaucracy is bad}. \\
\textit{Why this feels true}: when a person subjected to bureaucracy has a negative experience, the easiest attribution is to the least-understood aspect -- the bureaucracy.\\
\textit{What this is missing}: Bureaucracy itself is neither good nor bad. 

\ \\

\textit{Bureaucratic fallacy}: \textbf{Bureaucracy is a machine that has throughput and latency and dependencies and mechanisms. Bureaucrats are cogs in that machine}.\\
\textit{Why this feels true}: Complex large scale interlocking processes.
\textit{What this is missing}: Perceiving yourself as a cog in the wheel means a loss of agency

\ \\

\textit{Bureaucratic fallacy}: \textbf{Bureaucracy as a pure power struggle}. \\
\textit{Why this feels true}: \\
\textit{What this is missing}: 

\ \\

\textit{Bureaucratic fallacy}: \textbf{Bureaucracy as pure economic model}. \\
\textit{Why this feels true}: \\
\textit{What this is missing}: 

\ \\ 

\textit{Bureaucratic fallacy}: \textbf{product-focused narrative}. \\
\textit{Example}: \href{https://www.youtube.com/watch?v=OgVKvqTItto}{School House Rock's I'm Just A Bill}\\
\textit{Why this feels true}: \\
\textit{What this is missing}: 

\ \\

\textit{Bureaucratic fallacy}: \textbf{Bureaucracy is an aberration, a mistake, due to poor planning or incompetent participants}. \\
\textit{Why this feels true}: 
\textit{What this is missing}: 

\ \\

\textit{Bureaucratic fallacy}: \textbf{Bureaucracy is inefficient}. \\
\textit{Why this feels true}: Expressed by both subjects and bureaucrats who observe seemingly wasteful processes.
\textit{What this is missing}: If bureaucracy were truly inefficient (not allocating resources in the most efficient way), then it would be replaced by a more efficient approach. The key is to ask, ``efficient with respect to what metric?'' The metric of money, time, number of people, stability, robustness to perturbation.  Second, what would motivate improved efficiency? Without incentives, change is less likely. 

\ \\

\textit{Bureaucratic fallacy}: \textbf{Bureaucracy is due to malfeasance. The specific number of malicious bureaucrats ranges from ``all of the participants'' to ``just enough to be problematic.''} \\
\textit{Why this feels true}: There are bad actors. 
\textit{What this is missing}: Bureaucracy is unavoidable, and most of the participants are earnestly trying to help or are not making a positive contribution. Processes within bureaucracy are used to deal with malicious bureaucrats, like isolation or promotion. 

\ \\

\textit{Bureaucratic fallacy}: \textbf{Organizations are composed of individuals with personalities, and the inefficiency is attributable to a mashing together of distinct individuals with conflicting desires.} \\
\textit{Why this feels true}: It is true! (It isn't complete.) \\
\textit{What this is missing}: Other aspects like history of the processes (legacy) and protection against malicious subjects and protection against malicious bureaucrats. Analysis that stops at personalities misses emergent phenomena. Analysis that stops at personalities of individuals typically explains larger scale phenomena using personification of teams and organizations. 


\textit{Bureaucratic fallacy}: \textbf{subject-focused narrative}. \\
\textit{Why this feels true}: The person experiencing bureaucracy as a subject is confused by ``why isn't this easier?''  \\
\textit{What this is missing}: History of the processes (legacy) and protection against malicious subjects and protection against malicious bureaucrats. 

\ \\

\textit{Bureaucratic fallacy}: \textbf{Bureaucracy is a sign of decay from within the org.} \\
\textit{Why this feels true}: \\
\textit{What this is missing}: Bureaucracy is unavoidable emergence in any/every organization.

\ \\

\textit{Bureaucratic fallacy}: \textbf{system as an entity -- personification}. \\
See \cite{2002_Gall}
\textit{Why this feels true}: \\
\textit{What this is missing}: The problem with treating an org as entity is that apparent behavior is counter-intuitive. But when you break it into individual people their motives make more sense. This merely improve the post hoc rationalization. 

\ \\

\textit{Bureaucratic Fallacy}: \textbf{duration of a task is how long it would take one person to accomplish}.  \\
\textit{Why this feels true}: \\
\textit{What this is missing}: Fails to account for the overhead of interaction and delays due to asynchronous engagement.

\ \\

\textit{Bureaucratic Fallacy}: \textbf{people learn from their mistakes}. \\
\textit{Why this feels true}: \\
\textit{What this is missing}: Require a low latency feedback loop and incentive to change

\ \\

\textit{Bureaucratic Fallacy}: \textbf{processes are serial}.\\
\textit{Why this feels true}: \\
\textit{What this is missing}: 


\ \\

\textit{Bureaucratic Fallacy}: \textbf{Hard work creates results}.\\
\textit{Why this feels true}: \\
\textit{What this is missing}: 


\ \\

\textit{Bureaucratic Fallacy}: \textbf{Motivations for bureaucrats are categorized as individualistic, tribal, organizational, societal, or humanity}.\\
\textit{Why this feels true}: \\
\textit{What this is missing}: 


\ \\

\textit{Bureaucratic Fallacy}: \textbf{you cannot pay a little and get a lot}; see \href{https://en.wikipedia.org/wiki/Common_law_of_business_balance}{Common law of business balance}. \\
\textit{Why this feels true}: \\
\textit{What this is missing}: This doesn't allow for creative solutions and ignores \href{https://en.wikipedia.org/wiki/Nudge_theory}{nudge theory} from behavioral economics. 

