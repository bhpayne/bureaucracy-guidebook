\subsection{Written Communication\label{sec:written-communication}}

Reports, memos, emails are artifacts of bureaucracy in an organization. A written record creates evidence about policies and decisions. Existence of a record can be used for good or for harm.


% https://graphthinking.blogspot.com/2020/09/identifying-and-eliminating.html




%***************************

\subsection*{Why written communication does not happen\label{sec:written-comm-does-not-happen}}
\begin{itemize}
    \item Too many reports/emails/memos/messages to read and process and respond to. Emotionally overwhelmed.
\item The person may read slowly.
\item The person may type slowly, or their hand writing is poor.
\item Fear of imperfect communication. What if the email is incomplete or inaccurate or ambiguous?
\item The person views written communication as ``official" or ``plan of record'' and does not feel comfortable brainstorming or creating contingency plans
\item The person wants to avoid accountability for their statements.
\item The person may not be confident in their writing ability -- spelling, grammar, sentence composition, structuring content. \footnote{For tips on writing, see 
\href{https://en.wikipedia.org/wiki/The_Elements_of_Style}{The Elements of Style}
and
\href{https://www.google.com/search?q=dodm+5110.04}{DoDM 5110.04, Manual for Written Material}}
\end{itemize}



%***************************

\subsection*{Email: A Piece of Art, A Form, and A Game\label{sec:art-form-game}}

Originality is not a requirement in bureaucratic writing. Plagiarism is acceptable. The consistency of madlibs-based forms that follow decision trees is efficient. 

Readers need both context and conciseness. 

Using visual cues to improve readability,

Email is a game of documenting decisions and responses so that the sequence of interactions is clear for audits and recrimination. 

\subsection*{Email Responsiveness\label{sec:email-responsiveness}}

There are three tiers of email responsiveness:
\begin{enumerate}
    \item You are able to read all emails and reply to all emails.
    \item You are able to read all emails and reply to some.
    \item You are unable to read all incoming emails. 
\end{enumerate}
The following scenarios are focused on case 2, where you are able to triage (skim) emails but not enough time to respond to all.


Given insufficient time, which email do you reply to? An email from your boss, an email from your peer, an email from a person who reports to you, or a person who you do not know?
(Also, the ratio of these emails is not one to one to one to one.)

The email from your boss is likely the top priority. Of the remaining emails (peer, subordinate, unknown), the peer email is likely the next priority.
The subordinate and unknown person are likely last.

When there's insufficient time available, your transparency to subordinates and unknown people is likely to decrease

****************************
Three emails come in. One has no action, one is easy to reply to, and the third is difficult and takes time. Which one gets the response?
The ratio of these emails is not one to one to one.

1) the email that is easy to reply to gets answered first
2) the email that is difficult is second

If you only have time for one of the emails, it's likely to be the easy one. 
\href{https://en.wikipedia.org/wiki/Ambiguity_aversion}{ambiguity aversion}

As a consequence, outsiders see this as you are bike-shedding.
\href{https://en.wikipedia.org/wiki/Law_of_triviality}{law of triviality}

