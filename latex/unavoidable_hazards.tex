\section{Unavoidable Hazards of Bureaucracy\label{sec:unavoidable_hazards}}

In a bureaucracy there are certain problems that cannot be avoided. The value in recognizing them is to understand that what you're experiencing isn't an anomaly. The problem isn't unique to you, your circumstances, your coworkers, or the organization. The cause is the combination of those.

\ \\

\textbf{Separation of responsibility and accountability}. \\
Each bureaucrat in an organization has responsibilities associated with their role. The ability to complete the tasks associated the role are not wholly within the scope of the bureaucrat's control. Even if there is a desire for action, the action might not be immediately feasible because of a dependence on another person or process. 

\ \\

\textbf{Appreciation for being an effective bureaucrat is rare.}\\
If you do your job well, at best no one will notice.

``No one [thanks teachers] for policing cheating. Not the cheaters, not the honest students who feel inconvenienced and mistrusted, and certainly not the school [administrators] who have to process academic dishonesty paperwork.''\footnote{https://dynomight.net/teaching/}

The number of thank you cards sent to \href{https://www.fda.gov/}{Food and Drug Administration} meat inspectors, \href{https://www.osha.gov/}{Occupational Safety and Health Administration} regulators, \href{https://www.fcc.gov/}{Federal Communications Commission}, \href{https://www.ftc.gov/}{Federal Trade Commission}, and \href{https://www.sec.gov/}{Securities and Exchange Commission} is likely small. 
% TODO: ask each agency how many thank you letters they receive

There are counter examples in public service bureaucracy. 
Law enforcement is thanked when there is a victim of a crime. The military is held in high regard. 

\ \\

\textbf{Defining success is subjective and dynamic}. \\
Who defines success and for which audience in a bureaucratic organization is subjective because of the lack of feedback mechanisms (consequences are not immediately obvious). Worse, how success is defined can be changed at any time -- there's no need for consistency. 

\ \\

\textbf{Change threatens incumbents}. \\
Change of plans, roles, tasks, resources, flatness of org, scope, technology 

\ \\

\textbf{Diffusion of responsibility in the bureaucracy} \\
A specific task needs to be completed, and action requires involvement of multiple collaborators. 

\ \\

\textbf{Diffusion of blame in the bureaucracy}. \\

\ \\

\textbf{High latency feedback}. \\

\ \\

\textbf{Weak feedback}. \\

\ \\

\textbf{The person making the rules that you follow doesn't actually know what they're doing}. \\

Choices then include
* follow the rules that are not correct. Decreases productivity and morale
or 
* you can violate the rules and be more effective 
or 
* you can work the change the rules (and then are not doing the work that's needed)


\ \\

\textbf{Rarely get to pick who is on the team}. \\
When a task requires collaboration, there is rarely a choice of who you get to work with. 

\ \\

\textbf{Rarely able to alter team membership}

\ \\

\textbf{inadequate resources: staffing, time, money}

\ \\

\textbf{outcomes for the team are ill-defined and constantly shifting}

\ \\

\textbf{the organization has a lack of vision; or has vision but no plan; or has vision and plan but no consensus.}

\ \\

\textbf{progress depends on subjective decision making}

\ \\


\textbf{Bureaucracy is inefficient}\\
Different participants have different motives, and the aggregation needed for coordination is inefficient regardless of what metric efficiency is measured against.

Another example: Inefficiency of changing the requirements on a project partway through. If an objective quantitative measure were available, the ROI could be determined. 

\ \\

% https://graphthinking.blogspot.com/2020/07/scope-creep-is-experienced-differently.html
\textbf{Scope creep} \\
The customer wants more, and the exploration of what's possible is exciting (a positive experience).

The creator hears more work. This implies a few trade-off options, all of which are negative for one or both parties.
\begin{itemize}
    \item sticking with the original terms (telling the customer "no", which is negative for both parties)
    \item re-negotiation for additional compensation (a burden to both parties)
    \item the programmer doing more for the same pay, which means less money per effort (yielding a less happy programmer)
    \item decreasing existing efforts to fit the additional new requirements (yielding a less happy customer)
    \item even if the programmer is getting paid by the hour, additional work means the end product will be delayed to accommodate additional features (yielding a less happy customer)
\end{itemize}

\ \\

% https://graphthinking.blogspot.com/2020/09/why-migrating-from-current-to-new.html
\textbf{Migrating technologies} \\
The person implementing the transition has to be educated in both the old and new technology. 
the legacy code has to be migrated to the new implementation
convincing stakeholders; may require synchronization
difficulty scales with the number of stakeholders 