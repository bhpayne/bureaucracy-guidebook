\section{Unavoidable Hazards of Bureaucracy\label{sec:unavoidable_hazards}}

In a bureaucracy there are certain aspects which will be problematic and cannot be avoided. The value in recognizing them is to understand that there isn't a local anomaly. The cause isn't you, the people you with, or the organization you are a bureaucrat in. 

\ \\

\textbf{responsibility without authority}

\ \\

\textbf{very difficult to define success}. Diffusion of responsibility is symmetric with assigning blame. Feedback delay is often high.

\ \\

\textbf{change threatens incumbents}. Change of plans, roles, tasks, resources, flatness of org, scope, technology 

\ \\

\textbf{Diffusion of blame in the bureaucracy}

\ \\

\textbf{Rarely get to pick who is on the team}

\ \\

\textbf{Rarely able to alter team membership}

\ \\

\textbf{inadequate resources: staffing, time, money}

\ \\

\textbf{outcomes for the team are ill-defined and constantly shifting}

\ \\

\textbf{the organization has a lack of vision; or has vision but no plan; or has vision and plan but no consensus.}

\ \\

\textbf{progress depends on subjective decision making}

\ \\


\textbf{Bureaucracy is inefficient}\\
Different participants have different motives, and the aggregation needed for coordination is inefficient regardless of what metric efficiency is measured against.

Another example: Inefficiency of changing the requirements on a project partway through. If an objective quantitative measure were available, the ROI could be determined. 

\ \\

% https://graphthinking.blogspot.com/2020/07/scope-creep-is-experienced-differently.html
\textbf{Scope creep} \\
The customer wants more, and the exploration of what's possible is exciting (a positive experience).

The creator hears more work. This implies a few trade-off options, all of which are negative for one or both parties.
\begin{itemize}
    \item sticking with the original terms (telling the customer "no", which is negative for both parties)
    \item re-negotiation for additional compensation (a burden to both parties)
    \item the programmer doing more for the same pay, which means less money per effort (yielding a less happy programmer)
    \item decreasing existing efforts to fit the additional new requirements (yielding a less happy customer)
    \item even if the programmer is getting paid by the hour, additional work means the end product will be delayed to accommodate additional features (yielding a less happy customer)
\end{itemize}

\ \\

% https://graphthinking.blogspot.com/2020/09/why-migrating-from-current-to-new.html
\textbf{Migrating technologies} \\
The person implementing the transition has to be educated in both the old and new technology. 
the legacy code has to be migrated to the new implementation
convincing stakeholders; may require synchronization
difficulty scales with the number of stakeholders 