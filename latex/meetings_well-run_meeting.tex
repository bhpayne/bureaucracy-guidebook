\subsection{Well-run meeting\label{well-run_meeting}}

Explaining how to run an effective meeting (the scope of this section) is easy. Explaining why meetings are ineffective is a \href{https://en.wikipedia.org/wiki/Nash_equilibrium}{Nash equilibrium}. No one player benefits by reducing the number of meetings or effectiveness even though the group would benefit. So we all sit in ineffective meetings. 

The scope of this section is distinct from the processes covered by \href{https://en.wikipedia.org/wiki/Robert\%27s_Rules_of_Order}{Robert's Rules of Order}\footnote{I have not used Robert's Rules of Order for a meeting internal to a bureaucracy. Inside bureaucracies I'm aware of Robert's Rules of Order does not appear to be commonly used.}. 

\subsubsection{Form relationships and understand constituents before the Meeting}

\href{https://en.wikipedia.org/wiki/Nemawashi}{Nemawashi}

\subsubsection{Don't invite everyone}
Identify essential attendees. If someone does not need to be present, notify them in advance that you will share the meeting notes afterwards. 

\subsubsection{Create and use an Agenda}
\textit{Bad}: no meeting agenda\\
\textit{Good}: Have an agenda. \\
\textit{Better}: \underline{Share the agenda with other participants}. Having an agenda keeps attendees focused.  Enables tracking of progress during the meeting so participants are more likely to get to all topics.
\textit{Best}: For formal meetings, \underline{share agenda in writing prior to meeting}. Sharing the agenda in advance allows attendees to prepare.

Forces conspiring against agendas: takes time to create an agenda. Attendees might not take the time to read beforehand. Attendees may not stick to the agenda during the meeting.

\subsubsection{Ensure facilities are adequate}
For formal in-person meetings, Verify meeting venue has sufficient space, seating, working IT equipment

For formal virtual meetings, ensure participants are familiar with virtual meeting controls

TODO: why logistics/infrastructure matter in a bureaucracy:

TODO: forces conspiring against logistics/infrastructure

Fire alarm or other emergencies. 

\subsubsection{Body language matters}

If you are not speaking, are you reclined or leaning forward on the edge of your seat? Are you looking at the speaker?

If your eyes are closed, other people don't know if you're picturing something or falling asleep. 

If you approve of something but don't want to verbally interject, a thumbs up is useful signaling. 

\subsubsection{Take and share Meeting Notes}

Meeting notes are more detailed than the agenda but less detailed than a transcript of who said what. Meeting notes synthesize the discussion. Meeting notes specify follow-on who is taking which actions with what deadlines. 

For some insight on why meeting notes do not get taken, see \S\ref{sec:written-comm-does-not-happen}. 

\subsubsection{Facilitator ensures Presenters are Capable}

Does the presenter know how to project materials? How to present slides?

\subsubsection{Facilitator's ground rules}

To run a smooth and productive meeting, I explicitly state two ground rules to the attendees:
\begin{itemize}
    \item If you want to talk, raise your hand and I will call on you. If there are multiple people wanting to talk, I'll track the order of speakers.
    \item If you talk too long, I'll cut you off. 
\end{itemize}
This approach is critical when there are many people present, when people with diverse backgrounds are present, or when there is a mixture of dominant and submissive personalities present. 
If a visual signal like hand raising is not used, reliance on verbal interruption defaults to dominant personalities. Waiting for a person to finish speaking doesn't work for everyone because some participants will use more than their fair share of time. speaking for a long time needs to be addressed regardless of whether an intentional effort to exclude others or a consequence of verbosity.

As a facilitator, my focus is on structure (distinct phases of the discussion) and ensuring participation. I remove myself from taking part in the discussion.

\subsubsection{Facilitate Asking Dumb Questions without Feeling Intimidated}

Asking a question of an expert from a position of ignorance can feel intimidating. You may worry you're wasting the expert's time. In a recurring meeting a facilitator can address this by having participants write questions on paper and submitting anonymously. The questions or discussion topics can then be raised at subsequent meetings. 

To facilitate the anonymity every participant must be given paper and pen, and every participant must write something on the paper. The facilitator then has to collect the paper from each participant. For contributors who don't have a question, they can write down feedback about the meeting. 

This technique allows the expert to get the information needed for a response, or to figure out who the best person to respond is. 

\subsubsection{Collect feedback from Attendees on How to Improve}

Follow up with meeting attendees to get feedback on how to improve.