\section{Failure to Communicate\label{sec:failure-to-comm}}

To break the \href{https://en.wikipedia.org/wiki/Prisoner\%27s\_dilemma}{Prisoner's dilemma}, options are 
\begin{itemize}
    \item Expose all participants to the consequence of outcomes. In practice this feels unfair to each participant because the outcome is partially attributable to other people involved in the process. Dividing responsibility limits exposure to consequences.
    \item Have all participants communicate. In practice communication takes time and skill. Not everyone is willing to invest since communication is not seen as ``doing the work.'' Accounting for the \href{https://en.wikipedia.org/wiki/Allen\_curve}{Allen curve} takes effort. The time needed to arrive through consensus at an optimal approach for a given situation may exceed time available for solving the problem.
    \item Limit everything to what can be accomplished by one person. This hero-based approach is limited to the attention-bandwidth of the person and their skills. As the complexity increases the necessary skills increase and the number of candidate heros decreases. Large organizations accomplish complicated tasks by leveraging diverse skillsets of teams of bureaucrats.

\end{itemize}

%How does communication among individuals fail?

 
 
\href{https://en.wikipedia.org/wiki/Allen_curve}{Allen curve}\marginpar{[Tag] Folk wisdom}: 
\index{folk wisdom!\href{https://en.wikipedia.org/wiki/Allen_curve}{Allen curve}}
``exponential drop in frequency of communication between engineers as the distance between them increases.''

\href{https://en.wikipedia.org/wiki/Wiio\%27s_laws}{Wiio's laws}\marginpar{[Tag] Folk wisdom}: 
\index{folk wisdom!\href{https://en.wikipedia.org/wiki/Wiio\%27s_laws}{Wiio's laws}}
``Communication usually fails, except by accident.''


Weekly activity reports push good news up the chain and obfuscate the nuanced risk analysis. 
Leadership decisions are not pushed down the chain.

Role of assumptions 


There are ``levels of enlightenment":
\begin{enumerate}
    \item I feel bad
    \item I can't do what I want
    \item I can't do what I want in the way I want
    \item There is a problem
    \item I have a solution
    \item I have an implementation
    \item I tried but my solution didn't work
    \item Life sucks but I get a pay check
    \item I quit (in hopes of being more successful somewhere else)
\end{enumerate}
