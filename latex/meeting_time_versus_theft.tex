% https://graphthinking.blogspot.com/2021/02/organizations-value-things-more-than.html
In large organizations, there can be significant bureaucracy associated with even small purchases. A multi-step review process may be incurred for a \$2000 acquisition.

Another measurement of value is that if an employee were to steal even \$200 worth of materials, the organization would likely punish that employee.


Those metrics apply to tangible goods, but not to people's time. Consider a meeting of 10 people and each person's cost is \$200 per hour. A wasted meeting is not unusual and certainly would not incur bureaucratic review processes. The cost to the organization is fiscally the same -- \$2000. Similarly, consider an employee who is late and causes a loss of productivity. Merely depriving the organization of \$200 worth of time is not punished in the same way theft is.

In fact, organizations default to meetings (even recurring meetings) rather than not meet. And being late to a meeting is accepted. 

We can debate the differences between theft of materials and theft of time. The financial argument is clear. 

Source: Andy Grove in "High Output Management"