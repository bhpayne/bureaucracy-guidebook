\subsection{Building, Managing, and Spending Reputation\label{sec:reputation}}

Bureaucrats want to be all things to all people. Limiting your scope means disappointing some people. The team you are on may have a limited scope, but that is constrained only by the creative negotiation with other teams in the organization. Teams compete for budget, staffing, and glory. The budget and staffing are finite investments for improving reputation. 

/ //

What you as a bureaucrat can risk:
* your time
* your reputation
* your career
* your membership in the team or organization
* your self-respect
* the organization's budget
* the organization's staff
* potential for bonuses
* \href{https://en.wikipedia.org/wiki/Opportunity_cost}{opportunity cost}

Things I'm not willing to risk:
* my integrity
* my health (physical, mental, emotional)

/ //

techniques for building social capital: \S\ref{sec:credit-others} and \S\ref{sec:take-blame}

/ //

Managing reputation means acknowledging that your interaction with others is partially performative.

Your personal reputation within the organization dramatically impacts your effectiveness.

Organizations have reputations externally. 
Internal-to-the-org there is cultural norms. 
% https://graphthinking.blogspot.com/2021/01/why-active-shaping-of-culture-is.html


% https://graphthinking.blogspot.com/2018/05/my-evolving-view-on-role-of-my.html


Building reputation through multiple small wins or larger risk on bigger bets


Relation between Reputation and Brand and Political Capital? Same thing?

% the following article is useless
% https://www.indeed.com/career-advice/career-development/build-a-reputation
% since it reduces to "be a good person"

How does an individual create and accumulate political capital? What does political capital mean with respect to teams?

% https://graphthinking.blogspot.com/2021/09/notes-from-class-on-being-politically.html

Reputation matters for influence. How other people perceive you impacts what you can accomplish and when people seek out your help or input.

Your reputation is actively changing based on your activities and communication -- both your communication and the stories others tell about you.

Neglecting to manage your reputation means you lose input to the stories others tell about you. Active management of your reputation requires engaging people and generating evidence. 

Reputation is set whenever and where ever you are observed, or artifacts are associated with you. What you wear, when you show up, how your emails appear, body posture in meetings. 


Reputation is perception of the person the bureaucrat is engaging with.  What does that person think of you?

Ideally that would be a function of their technical skill, ability to collaborate with other people, the strength of their network, creativity. None of this matters if the person you're engaging with doesn't know those things. 

Based on your reputation, what trust does that person have? 

To spend reputation is to bend the rules. 

Spending reputation means taking risk that involve other people

Build reputation by doing useful things that are visible to other people
