\subsection{Building, Managing, and Spending Reputation\label{sec:reputation}}

%How does an individual create and accumulate political capital? What does political capital mean with respect to teams?

\subsubsection{Relevance of Reputation to Bureaucracy}

As a bureaucrat you may want to be all things to all people. Since that is not feasible, you have to limit your scope and therefore disappoint some people. This same dynamic of a limited scope applies to the team you are on, and again to the organization the team is part of.

The good news is that you have influence over how others perceive your limitations. Creative negotiation with other bureaucrats, other teams, and other organizations shape their perception of your value. 

\subsubsection{Definition of Reputation}

Your reputation as a bureaucrat is what other people expect from you. This section is about actions you can take to create a reputation and how to spend political capital. 



Reputation is perception (What does that person think of you? your team? your organization?). Reputation is set whenever and where ever you are observed, or artifacts are associated with you. What you wear matters, when you show up impacts your image, how your written communication is read alters perceptions, and your body posture in meetings informs other people. 



\subsubsection{Reputation, Brand, Image}

There are multiple phrases that all refer to the same concept of being perceived and the associated expectations. Individuals create (or get) a reputation; organizations have brands. A team of bureaucrats can have a reputation. 

\subsubsection{Your Reputation matters}

Your personal reputation within the organization dramatically impacts your effectiveness. People will let you do things (or prevent you from doing things) based on their expectations about you. 

Your reputation matters for influence: whether other people turn to you depends on what they expect from you. 

How other people perceive you impacts what you can accomplish and when people seek out your help or input.

\subsubsection{You can Manage your Reputation}

Managing reputation means acknowledging that your interaction with others is partially performative. This may feel disappointing if you want to be judged solely on your productivity or knowledge. 

Your success is limited if you focus exclusively on doing the work, and your success is limited if you focus exclusively on the performative aspects. 
Neglecting to manage your reputation means you lose input to the stories others tell about you. Active management of your reputation requires engaging people and generating evidence. 

Your reputation is dynamically changing based on your activities and communication -- both your communication and the stories others tell about you.


\subsubsection{Techniques for building political capital}

Ideally reputation would be a function of your technical skill, ability to collaborate with other people, the strength of your network, creativity. None of this matters if the person you're engaging with doesn't know those things. 


Because the definition of reputation is about expectations other people have about you, what you choose to work on matters. How you work on your tasks (creativity, enthusiasm, dedication), the artifacts produced, and your ability to communicate all shape your reputation. 

Building reputation through multiple small wins or larger risk on bigger bets

All that also applies to teams of bureaucrats. Additionally, teams compete for budget, staffing, and glory. How the team's budget and staffing are spent impact reputation. 


Give credit to others (\S\ref{sec:credit-others}) and offer to take blame (\S\ref{sec:take-blame}).

\subsubsection{Using Good News (or Early News to build Influence}
Rather than tell good news directly to a top decision maker, first inform the person who influences the decision maker. Tell the influencer the information does not need to be credited to me.

Benefits:
\begin{itemize}
    \item Improves the influencer's reputation with the decision maker.
    \item The influencer has the chance to contextualize the information for the decision maker.
    \item The influencer has the option to leverage the information in ways I would not have.
    \item My reputation with the influencer is enhanced.
    \item I reaffirmed the influencer's role and status.
    \item Influencers are easier to access.
\end{itemize}

\subsubsection{How to spend your reputation, and your team's reputation}
Based on your reputation, what trust does that person have? 

Social Capital, Political Capital

What you as a bureaucrat can risk:
* your time
* your reputation
* your career
* your membership in the team or organization
* your self-respect
* potential for bonuses
* the organization's budget
* the organization's staff
* \href{https://en.wikipedia.org/wiki/Opportunity_cost}{opportunity cost}

Things I'm not willing to risk:
* my integrity
* my health (physical, mental, emotional)


Whenever you are engaging within your team, you are either actively spending or building your reputation within the team.
Whenever you are engaging with people from outside your team, you are either actively spending or building your teams reputation

Ask for a favors and provide no value
Explorer options that other people don't see is worthwhile and then there's no payoff
Produce nothing of value to the organization or other people
Talk with people outside your team and misrepresent the efforts of your team
Speaking honestly about the faults of your own team to people outside your team is also harmful

To spend reputation is to bend the rules. 

Spending reputation means taking risk that involve other people

Build reputation by doing useful things that are visible to other people




%Internal-to-the-org there is cultural norms. 
% https://graphthinking.blogspot.com/2021/01/why-active-shaping-of-culture-is.html


% https://graphthinking.blogspot.com/2018/05/my-evolving-view-on-role-of-my.html


% the following article is useless
% https://www.indeed.com/career-advice/career-development/build-a-reputation
% since it reduces to "be a good person"


% https://graphthinking.blogspot.com/2021/09/notes-from-class-on-being-politically.html




