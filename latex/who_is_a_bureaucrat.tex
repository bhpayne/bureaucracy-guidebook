\subsection{Who is a bureaucrat?}

A cashier in a gas station is a bureaucrat. The ``policy'' might simply be ``take money from customer in exchange for items and gas,'' but the subjective application of that policy leaves a lot of room for the cashier to shape the customer's experience. Does the cashier greet the customer when the customer enters the store? Does the cashier look at the customer to acknowledge the customer? Smile? How quickly does the cashier engage the customer? Minor nuances that are left to the cashier in the execution of the store policy means there is room for subjective application of the policy. 

% examples of bureaucrats
A bank teller, a loan officer, and a bank's \href{https://en.wikipedia.org/wiki/Technical_support}{technical support} are all bureaucrats. Each person subjectively enforces policies on behalf of the organization. The directness of financial impact for the person varies among these roles. Of these three roles, the loan officer's interactions with bank customers provides the clearest feedback on profit. The loan officer doesn't act alone though -- the customer's interactions with tellers and the bank's technical systems also matter to the customer's decision. 


This same discretionary application of policy applies to commercial bureaucrats like sandwich makers, car salespeople, oil well drillers, grocery clerks, retail clerks, and plumbers. Public school teachers, state and federal police, military members, tax collectors, and other state workers are government bureaucrats. 


% bureaucracy is not limited to white collard office workers
Factory line workers subjectively apply policies. Enforcement of quality standards is the most obvious area. Pacing of work is a negotiation with management that directly impacts productivity and profits.

Structured environments like sports featuring well defined rules do not eliminate bureaucracy. Teammates use subjective policies on who to work with and how to best leverage their strengths and exploit the opponent's weaknesses. The policies are set in part by the coach. Referees make subjective determinations about rules.

Sometimes bureaucrats do not work directly with customers or citizens or products. Then the bureaucratic process is inflicted on fellow bureaucrats. In this scenario, a bureaucrat is subjectively applying a policy to other bureaucrats. 

Identifying yourself as a bureaucrat matters, both to the employee and to the business. The risk of not self-identifying as a bureaucrat is that you won't grasp how much control you have in implementing and enforcing policy. If you think of yourself as having to blindly follow rules, you will harm the people you are applying the rules to and you will harm the business/institution you are applying the rules for. Adapting policies to circumstances is the value of having judgement capacity. 

In a similar sense from the consumer/citizen perspective, if you don't think you are interacting with a bureaucracy, you won't perceive the opportunity to negotiate.  If you view rules as fixed and inflexible, you will harm your ability to make progress. If a rule was made by a human, then that rule is flexible. Who made the rule? Who enforces the rule? If you can talk to them, could they be convinced to make a modification or an exception?

If you don't think about a bureaucratic framing, you might think the store clerk is enforcing a policy because they don't like you. Assigning personality conflict as the cause might lead to a different conversation with their manager (the person who created the policy). 

If you don't think of yourself as a bureaucrat, you'll behave differently in your job. The paradigm of ``just tell me what to do'' is the default (learned in school) and you won't know how to engage with coworkers/bosses since they are not friends. You will be less likely to understand how to leverage the organization and identify collective wisdom. Thinking from the perspective of a bureaucrat explains why evangelizing within your organization is relevant. 
%You'll see your coworkers as competitors for promotion.

% https://graphthinking.blogspot.com/2020/10/impact-of-self-identifying-as-not.html
If you think of yourself as merely a cog in a machine, you are less likely to notice that you exert influence in the process and you are less likely to recognize the autonomy available to you. 
If you think ``I have a real job (e.g., nurse, cashier, teacher), therefore I'm not a bureaucrat", you are less likely to recognize the subjective power you have in interactions with the public.
If you think, ``I'm at the bottom of my organization's hierarchy, therefore I do not have power", you are less likely to notice the autonomy when it is available.

If you don't think of yourself as part of a bureaucratic process, you'll behave differently in interactions with bureaucrats.  You won't perceive opportunities to negotiate because the processes seem fixed instead of subjective. 
You won't recognize motives and incentives of bureaucrats, so their activities will seem incomprehensible.


