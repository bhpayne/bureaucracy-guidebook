\section{Not many options for organizations}
% https://graphthinking.blogspot.com/2021/07/patterns-anti-patterns-in-bureaucracy.html

Organizations comprised of bureaucrats have fewer options than individual bureaucrats. The choices an individual bureaucrat faces are described in the section on dilemmas, \S\ref{sec:dilemma_trilemma}. In comparison, the choices faced in the design of an organization are
\begin{itemize}
    \item flatness of organizational hierarchy
    \item number of supervisors per employee
    \item teams: create a new team, merge existing teams, dissolve a team
    \item processes: hiring, promotion (pay or title), professional training, firing
\end{itemize}
In a government bureaucracy, pay and financial incentives are typically set outside the organization.

The organization may have policies and processes regarding hiring, promotion, training, and firing, but the decision may not be made by the top-level management. 

\subsubsection{Not many options for Teams}

Team managers decide who gets hired and who gets promoted and who goes to what training and who gets fired.

Among teams, interactions are either lateral (sideways) or parent-child (top-down) or child-parent (bottom-up) \cite{2014_Jorgensen}.

The child-parent bottom-up upward communication is either inadequate (too few updates, or not enough information, or insufficient context), relevant, or excessive. Finding the balance depends on the presenter knowing the individual audience members so that a tailored message is provided, and adapting to the specifics of the situation. A weekly or monthly report to multiple superiors is inadequate. 
% tips on managing up: 
% https://svpg.com/managing-up/

The parent-child downward communication either is inadequate (no direction), provides actionable vision, or micromanagement. 

For lateral interactions among teams, the tension between cooperation and competition manifest in struggles over
\begin{itemize}
    \item money
    \item staffing
    \item prestige
    \item products (output)
    \item resources (inputs)
    \begin{itemize}
        \item access to or control of data
        \item technology resources
        \item hardware
        \item floor space
        \item expertise
    \end{itemize}
\end{itemize}


