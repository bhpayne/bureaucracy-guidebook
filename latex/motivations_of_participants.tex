\subsection{Motivation of Bureaucrats}

To understand the bureaucrats you work with, appreciating their diverse motives is instructive. If you expect everyone to have the same motives as you, then you will be surprised by the friction. 

Motivations of participants are rarely ``how can I make the company more successful" or even ``how can I sell/produce more product"? Usually motivation is based on personal success in various manifestations, which leads to emergent phenomena which appears confounding to observers outside the bureaucracy. 


% see https://en.wikipedia.org/wiki/Social_influence

Each bureaucrat has a motive, even the bureaucrats who do nothing. 
% https://graphthinking.blogspot.com/2020/02/there-is-no-idle-status-for-paid.html
In an organization where you are a paid bureaucrat, you are either actively working for improvement of the organization, or your existence is parasitic to the organization. There is no ``idle" status for paid employees in an organization with limited resources.

Not too efficient such that I eliminate the need for my job, and not so inefficient that the organization fails and I lose my job. Increasing the efficiency of bureaucracy is good for the organization and the outcomes, but can be harmful to the bureaucrat's career.

Career stability within an organization is a benefit, and it can be leveraged to take more risk. However, it typically manifests as inaction by an employee. There's no harm to the employee in not taking action. If an employee doesn't do anything, nothing bad will happen to that employee. Career stability decreases extrinsic motivation.



Example motivations for bureaucrats: stability, money, travel, problem solving, like being associated with org, logistical convenience (``the office is near where I lived.'')

