\section{Motivation of Bureaucrats\label{sec:motivations}}

Appreciating the diverse motives of the bureaucrats you work with is instructive. If you expect everyone to have the same motives as you then you will be surprised by the friction created by diverse motives. 

Motivations of participants are rarely ``how can I make the organization more successful" or even ``how can I sell/produce more product"? Usually motivation is based on personal success in various manifestations, which leads to emergent phenomena which appears confounding to observers outside the bureaucracy. 


% see https://en.wikipedia.org/wiki/Social_influence

Each bureaucrat has a motive, even the bureaucrats who do nothing. 
% https://graphthinking.blogspot.com/2020/02/there-is-no-idle-status-for-paid.html
In an organization where you are a paid bureaucrat, you are either actively working for improvement of the organization, or your existence is parasitic to the organization. There is no ``idle" status for paid employees in an organization with limited resources.

As an example motive for a bureaucrat, I want to avoid being too efficient such that I eliminate the need for my job, and not so inefficient that the organization fails and I lose my job. Increasing the efficiency of bureaucracy is good for the organization and the outcomes, but can be harmful to the bureaucrat's career.

Career stability within an organization is a benefit, and it can be leveraged to take more risk. However, it typically manifests as inaction by an employee. There's no harm to the employee in not taking action. If an employee doesn't do anything, nothing bad will happen to that employee. Career stability decreases extrinsic motivation.


Example motivations for bureaucrats: 
stability (aka job security, the comfort of a routine),
money (current pay or future earnings), 
travel, 
problem solving, 
status, 
exerting power or control, 
credibility of being associated with the organization (if the organization has a positive reputation), 
logistical convenience (``the office is near where I lived''), 
service to people the organization serves.



The consequence of diverse motives is that expecting bureaucratic organizations to be logical, fair, consistent, and efficient is unreasonable even when every participant wants those features. Each bureaucrat thinks, ``I am logical, fair, consistent, and efficient.'' Therefore each bureaucrat expects other bureaucrats to meet those same (unrealistic) standards. Next, anthropomorphize the team or organization and expect the group to meet those standards. 

Even if each bureaucrat were logical, fair, consistent, and efficient (they are not, and neither are you), each person has a different motivation. Each person wants to accomplish something different using their unique skills and referencing their own experiences. Compounding the confusion, each bureaucrat has to coordinate using communication that has latency and limited bandwidth and isn't precise.

An expectation that bureaucracy feels illogical, unfair, inconsistent, and inefficient is a useful baseline. Working against bureaucratic entropy yields improvements even though perfection is inaccessible.