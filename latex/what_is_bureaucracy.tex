\section{What is bureaucracy?\label{sec:define_bureaucracy}}

While you may know it when you see or experience bureaucracy, for this book definitions are useful. 

Bureaucracy involves creation and execution of policies. Creating and carrying out policies usually involves multiple people, with each person having a specialized roles. Multiple people form an organization to address issues like scalability, complexity, or latency. The organization has control over the disbursement of resources relevant to the society the organization operates within, or administers a policy within that society. Resources managed by the organization are either tangible (e.g., water, air, land) or expertise.  

Bureaucracy is not limited to government. Non-profit organization, volunteer groups, commercial companies, and even small teams of people can invoke bureaucratic tendencies. The existence of bureaucracy is independent of an organization's purpose. Carrying out someone else's subjectively defined policy will require you to make your own subjective decisions regarding execution and enforcement. 

\ \\

With bureaucracy defined, there are distinct roles that can be identified.
A bureaucracy typically involves a policy creator, a policy enforcer, and the person upon whom policy is inflicted. In the context of government, the policy creator can be either a politician or a bureaucrat. 

A \gls{bureaucrat} is a person subjectively interpreting policies on behalf of an organization and has discretionary enforcement to facilitate coordination of stakeholders. 

Let's break that down piece-by-piece. First, ``subjective interpretation'' means there is a person making a decision about how to do something. The subjectivity arises from different reasons one might choose an option over a competing option.  ``Policies" is a set of actions in a given circumstance. ``An \gls{organization}" is the collection of people for who the policy is made. ``Discretionary enforcement'' means the person is choosing how to apply the policy in the specific circumstances. ``Facilitating coordination'' means bureaucracy is about getting multiple people (or sometimes a person at different instances in time) to work together. The ``stakeholders'' is a group of people who care about the application of the action in each circumstance.  That's still pretty dense, so the rest of the book is spent expanding on the nuances and implications of this definition.

Bureaucracy is neither good nor bad. Bureaucracy is not tied to politics, or any specific institution (corporations, governments, academics). Bureaucracy is not defined to be efficient nor, does it have to be inefficient. Bureaucracy is not restricted to paperwork, or record keeping, or quantification, or gathering metrics. 

Bureaucracy is about delegation of control, communication, decision making, coordination, and processes. Involves negotiation, primarily informal. 

An organization comprised of bureaucrats is a \gls{bureaucracy}. The definition of bureaucracy used in this book is independent of government. Nothing in this definition involves paperwork or an office building. Definitions that limit the concept of bureaucracy to specific contexts result in a decreased ability to describe complex large-scale human organizations. 

The protagonist within a \gls{bureaucracy} is the \gls{bureaucrat} -- the person who is a member of an organization and is responsible for subjective implementation of policy for the organization. The person that a bureaucrat's decisions are inflicted on a \gls{subject}.  Depending on context, a subject may be a student (when the bureaucrat is a teacher) or a subject may be a citizen if the bureaucrat is a police officer or government official. Sometimes a bureaucrat's decisions are inflicted on other bureaucrats-as-subjects, such as when a Chief of Police creates guidelines for police in their district, or when a senior diplomat sets policy for embassy employees. 

A critical aspect of bureaucracy is that everything is made up, specifically by other humans. The consequence is that everything is negotiable. You (in the role of either a subject or a bureaucrat) need to know who to negotiate with and how to negotiate the desired changes. The only actual rules are mathematical physics that describe nature. Everything else is either naturally occurring macroscopic emergent phenomena (e.g., chemistry, biology) or humans making up labels and norms. 

Bureaucracy arises when there is no common objectively quantifiable feedback mechanism for individual participants in the organization. This aspect is why governments, schools, and prisons are characterized as bureaucratic. The military doesn't rank soldiers by ``number of enemies killed'' and is bureaucratic. Even profit-driven commercial organizations are bureaucratic when the impacts of individual employees are not coupled to sales metrics. 

Profit-based feedback makes some roles in a business context slightly more predictable and understandable, though there are still trade-offs like long-term profit versus short-term profit and externalization of harm. 

The concept of bureaucracy is most visible for complex, long lasting, and recurring situations involving many people. The apparent friction can be lower when there are only a few people involved (``I'm just talking to my collaborator" or ``I'm just buying groceries from a clerk at the store'' or ``I'm using a website for a government service''), but there is a continuous gradient. 

There is the external resource (mail delivery for USPS, public safety for FBI, environment for EPA) and there are resources internal to the bureaucracy. The focus of this book is on internal resources. In that context, bureaucracy is for the disseminated responsibility for use of resources: attention, skill, expertise. Time, money, staffing are proxy measures.



A useful way to think about bureaucracy is as a system for distributed knowledge and distributed decision making. That is in contrast to easier-to-understand concepts like centralized knowledge and centralized decision making. A government run by dictatorship is easy to conceptualize compared to democracies because there is a central character around which a narrative can be formed. Similarly, telling stories about the \href{https://en.wikipedia.org/wiki/Chief_executive_officer}{CEO} of a company is much easier than capturing the thousands of interactions conducted by the many employees of that company. Linear story-telling with a small number of protagonists does not map well to the complexities of bureaucracy. 
% are there alternatives to Bureaucracy that accomplish the same non-centralized non-consensus approach to complexity?


% https://graphthinking.blogspot.com/2017/09/market-friction-and-bureaucratic.html
Distributed knowledge and distributed decision making are hindered by
\begin{itemize}
    \item limited bandwidth between people, specifically the bureaucrats involved
    \item non-zero latency of information between people, specifically the bureaucrats involved
    \item the cost of measurement (getting data)
    \item the cost of analysis of the data
    \item making decisions that are suboptimal
\end{itemize}

