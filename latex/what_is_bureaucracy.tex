\subsection{What is bureaucracy?}
While you may know it when you see it or experience it, for this book a definition is useful. 
\gls{Bureaucracy} is the subjective interpretation of policies and discretionary enforcement/implementation to facilitate coordination of stakeholders on behalf of an organization. 

Let's break that down piece-by-piece. First, ``subjective interpretation'' means there is a person making a decision about how to do something. ``Policies" is a set of actions for a given circumstance. ``Discretionary enforcement and implementation'' means the person is choosing how to apply the policy in the specific circumstances.  ``Facilitating coordination'' means bureaucracy is about getting multiple people (or sometimes a person at different instances in time) to work together. The ``stakeholders'' is a group of people who care about the application of the action in each circumstance. Finally, ``an \gls{organization}" is the collection of people for who the policy is made. That's still pretty dense, so the rest of the book is spent expanding on the nuances and implications of the definition.


Another way to think about bureaucracy is as a system for distributed knowledge and distributed decision making. That is in contrast to easier-to-understand concepts like centralized knowledge and centralized decision making. 
% are there alternatives to Bureaucracy that accomplish the same non-centralized non-consensus approach to complexity?

The concept of bureaucracy is most visible for complex, long lasting, and recurring situations involving many people. The apparent friction can be lower when there are only a few people involved (``I'm just talking to my collaborator" or ``I'm just buying groceries from a clerk at the store'' or ``I'm using a website for a government service.''), but there is a continuous gradient. 


The protagonist in a bureaucracy is the \gls{bureaucrat} -- the person responsible for subjective implementation of someone else's intent, with unquantifiable results. In this book the person that a bureaucrat's decisions are inflicted on a \gls{subject}. In various contexts, a subject may be a student (when the bureaucrat is a teacher) or a subject may be a citizen if the bureaucrat is a policeman or government official. Sometimes a bureaucrat's decisions are inflicted on other bureaucrats-as-subjects, such as when a Chief of Police creates guidelines for police in their district, or when a senior diplomat sets policy for embassy employees. 

