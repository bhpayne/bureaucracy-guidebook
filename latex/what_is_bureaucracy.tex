\section{What is bureaucracy?\label{sec:define_bureaucracy}}
While you may know it when you see it or experience it, for this book a definition is useful. 
A \gls{bureaucrat} is a person subjectively interpreting policies of an organization and discretionary enforcement/implementation to facilitate coordination of stakeholders on behalf of an organization. 

Let's break that down piece-by-piece. First, ``subjective interpretation'' means there is a person making a decision about how to do something. ``Policies" is a set of actions for a given circumstance. ``Discretionary enforcement and implementation'' means the person is choosing how to apply the policy in the specific circumstances.  ``Facilitating coordination'' means bureaucracy is about getting multiple people (or sometimes a person at different instances in time) to work together. The ``stakeholders'' is a group of people who care about the application of the action in each circumstance. Finally, ``an \gls{organization}" is the collection of people for who the policy is made. That's still pretty dense, so the rest of the book is spent expanding on the nuances and implications of this definition.



The definition of bureaucracy used in this book is independent of government. Nothing in this definition involves paperwork or an office building. Definitions that limit the concept of bureaucracy to specific contexts result in a  decreased ability to describe complex large-scale human organizations. 


Bureaucracy is emergent (requires sufficient scale) and arises when there is no common objectively quantifiable feedback mechanism for individual participants in the organization. The goals of the organization are not financial profit. 

The scale is important because there is no longer dependence on individual relationships (beyond \href{https://en.wikipedia.org/wiki/Dunbar\%27s_number}{Dunbar's number}. There are people in the organization that you don't know and for which there is no common accountability. An organization subdivided into team recursively until there is local person-to-person accountability.  

The local rules bureaucrats employ to enable distributed decisions using distributed knowledge is meetings, processes, and communications. 

Actually, bureaucracy is worse than emergent - the system rules can be altered or ignored by the stakeholders. \href{https://en.wikipedia.org/wiki/Wicked_problem}{Wicked problem}. This is why coming up with a holistic theory of bureaucracy is difficult. 

Commercial businesses have a different accountability -- money. Common across all participants within the organization, and common with external stakeholders. The goal of a company is to generate profit. Commercial businesses have people who make subjective decisions and enforce policies, but there is a common metric for feedback. The feedback mechanism is not perfect. Being a good commercial bureaucrat does not necessarily result in monetary success.

Prisons, schools, medical, government, military all consume and spend money, but money isn't the goal. When faced with a decision, choice is not guided by which will generate more profit. 


The concept of bureaucracy is most visible for complex, long lasting, and recurring situations involving many people. The apparent friction can be lower when there are only a few people involved (``I'm just talking to my collaborator" or ``I'm just buying groceries from a clerk at the store'' or ``I'm using a website for a government service''), but there is a continuous gradient. 


Another useful way to think about bureaucracy is as a system for distributed knowledge and distributed decision making. That is in contrast to easier-to-understand concepts like centralized knowledge and centralized decision making. A government run by dictatorship is easy to conceptualize compared to democracies because there is a central character around which a narrative can be formed. Similarly, telling stories about the \href{https://en.wikipedia.org/wiki/Chief_executive_officer}{CEO} of a company is much easier than capturing the thousands of interactions conducted by the many employees of that company. Linear story-telling with a small number of protagonists does not map well to the complexities of bureaucracy. 
% are there alternatives to Bureaucracy that accomplish the same non-centralized non-consensus approach to complexity?


The protagonist in a bureaucracy is the \gls{bureaucrat} -- the person who is a member of an organization and is responsible for subjective implementation of policy for the organization. In this book the person that a bureaucrat's decisions are inflicted on a \gls{subject}. In various contexts, a subject may be a student (when the bureaucrat is a teacher) or a subject may be a citizen if the bureaucrat is a police officer or government official. Sometimes a bureaucrat's decisions are inflicted on other bureaucrats-as-subjects, such as when a Chief of Police creates guidelines for police in their district, or when a senior diplomat sets policy for embassy employees. 

