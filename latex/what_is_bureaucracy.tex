\subsection{What is bureaucracy?}
While you may know it when you see it or experience it, for this book a definition is useful. 
\gls{bureaucracy} is the subjective implementation of policies to facilitate coordination of stakeholders. 

This concept is most visible for complex, long lasting, and recurring situations involving many people. The apparent friction can be lower when there are only a few people involved (``I'm just talking to my collaborator" or ``I'm just buying groceries from a clerk'' or ``I'm using a website for a government service.''), but there is a continuous gradient. 

Effectively, bureaucracy enables you to rely on other people by enabling other people. 