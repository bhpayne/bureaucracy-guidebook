\subsection{Creating change in the organization\label{sec:creating_change}}

If the organization you are in has no problems or challenges, this section can be skipped. For bureaucrats in organizations that do have issues, this part of the guidebook provides points to ponder independent of the specific problem.

As a bureaucrat, you have unique insight on the challenges the organization faces, and you have unique leverage to alter the situation.  While you could proceed haphazardly, an effective bureaucrat has vision, goals that break down the vision, plans on how to achieve each goal, and milestones which indicate whether the plan is proceeding. 

Perspectives to consider when assessing change include what the situation is, what the situation could be, and what the situation looks like from different stakeholders.

Confounding your ability to improve the organization, there are people around you who have conflicting visions or no vision. There are different views on whether something is actually a problem, different prioritizations, and different approaches to addressing problems.

A trade-off to consider is that having niche impact is easier than broad change. There's also a trade-off of the quick fix versus more robust solutions.

For a given structural problem in an organization, options include technical solutions, changing policy, or changing cultural norms of participants.

%Determine social/political/technical impediments. 

\textit{Tip}: Before starting a new effort, check to see whether this has been tackled before.\marginpar{[Tag] Actionable Advice}
Learn the history of the problem. Why hasn't this been solved?

\textit{Tip}: Query your first and second order social network.\marginpar{[Tag] Actionable Advice}

\textit{Tip}: Get feedback early \marginpar{[Tag] Actionable Advice} before polishing.

\textit{Tip}: Advertise the result.

\textit{Tip}: Hear criticism and respond.

\textit{Tip}: Leverage both social networks and bureaucratic processes. 

\textit{Tip}: Identify sources of power (hierarchical positions and titles, social influence, reputation and credibility, buzzphrases or popular paradigms) and leverage them.

\textit{Tip}: Professional respect (for what the other person knows) and professional curiosity (for what you don't know) \\
Example: Getting approval from multiple overseers in different hierarchies is hard. Often different objectives and incentives.

% https://graphthinking.blogspot.com/2016/01/methodology-for-people-acting-as.html
\textit{Tip}: Use social recommendations to identify relevant individuals.\\
Leverage the trust already in an existing social network by starting with ``Person A recommended I talk to you about X."

% https://graphthinking.blogspot.com/2016/01/methodology-for-people-acting-as.html
\textit{Tip}: Sit in on meetings, listen to topics, see who is talking, who is attending. After the meeting talk to individuals about the meeting. Set up one-on-one informal discussions. Keep the first conversation  brief - 10 or 15 minutes. Your body language should indicate engagement and interest. ``Who else would you recommend talking to?" is the last question in the first conversation.


\textit{Tip}: Consensus doesn't mean everyone agrees on the problem, the remedy, or the approach, or who's taking the action. Consensus in a bureaucracy means people aren't going to resist the change.