\section{Tips for How to Reduce Administrative Overhead}

You can read this list from a position of authority to take action, or you can read this list as an advocate for improvement. 

% https://graphthinking.blogspot.com/2018/08/how-to-decrease-bureacracy.html
In isolation, none of these ideas should be surprising.
\begin{itemize}
    \item Automate recurring decision processes. This decreases bureaucracy by removing subjective influence of bureaucrats
\item Where automation is infeasible, make customer advocates with end-to-end authority available. 
This decreases bureaucracy by improving customer's navigation of processes.
\item Make processes transparent to participants. 
This decreases bureaucracy by improving customer's understanding of processes.
\item Make information discoverable (e.g., via search engine) .
\item Make information directly available, rather than mediated by a person.
\item After an interaction is completed, summarize the steps and outcome for the participant. 
\item When a process fails the needs of a participant, investigate the failure and improve the process.
\item Make the goals and priorities of the organization clear to all.
\item Define measurable standards of performance, both for individuals and teams.
\item Train bureaucrats how to engage participants effectively; these interactions determine the culture.
\item Train bureaucrats by addressing their immediate problems (e.g., through mentorship).
\item Make employment desirable to people who have desirable characteristics (e.g., educated candidates).
\item Enhance accountability to peers (e.g., peer review of actions and outcomes).
\item Ensure that incentives for organizations and individuals encourage improvement rather than maintenance of status quo.
\item Decision making should be pushed down the hierarchy to the practitioner.
\item When decision making requires cross-organization interaction, form a team of practitioners.
\item Bosses should share workload with their team in order to gain practical exposure to current challenges.
\item Seek feedback from process participants; then provide status updates on the implementation process.
\item Replace sequential approval chains with a concurrent review process. Sequential approval reduces the number of ideas reaching later approvers by filtering good ideas as well as bad ideas. Approvers may perform independent (redundant) due diligence and end up disagreeing. 
\end{itemize}
