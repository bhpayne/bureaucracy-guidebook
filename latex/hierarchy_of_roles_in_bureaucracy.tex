\subsection{Hierarchy of Roles}
A major feature of bureaucracy is the \href{https://en.wikipedia.org/wiki/Organizational_structure}{structure of the organization} for which policies are created and carried out. Structure in an organization is dynamic, but at any point in time an organization typically has a defined set of roles. Those roles have different scopes of decision authority. 

Roles in an organization are defined by the boundaries of responsibility. The purpose of a role is to minimize conflict and reduce redundancy, allowing regularity and control. Clear responsibility enables effective bureaucracy. 

The choices in a hierarchy are 1) a supervisor oversees how many people and 2) a person has how many supervisors. Naively one might expect that an employee has one boss, but that is not a requirement. 
The more people a supervisor oversees, the flatter the organization. See \cite{2012_Valve} and \cite{1972_Joreen}.

The consequence of hierarchy in an organization means that as a member of the bureaucracy you do not have full autonomy -- otherwise you would not be a member of the hierarchy. At the same time you are not under strict control of the organization -- you still have some subjective decision making authority as a bureaucrat.

The person at the top of the hierarchy does not know everything. The person at the top of the hierarchy does not have input on every decision made in the organization. Some autonomy is retained by all members of the bureaucracy.

Independent of defined roles and designated titles in an organization's hierarchy, there are a set of implicit roles and a separate social hierarchy of informal influencers and decision makers. Informal influencers in a bureaucracy usually have long relationships with the decision maker or relevant credentials or both. The credentials can be formal (e.g., a PhD) or informal (demonstrated success on a project). In either case, the decision maker is relying on another person's expertise. 

Another set of informal relations within an organization is mentors and mentees. These relations allow mentors to transmit institutional knowledge to mentees, and allows people in senior positions to access the novice perspective. 


\ \\
One consequence of hierarchy is a sense of fear felt by people who report to other people because they think they are disempowered. A person, Sue, is perceived to have power over another person, Amy, because Amy gave up some control to Sue. Amy not having control creates fear, regardless of how Sue behaves.   


% Active bystander when the person doing wrong is in a position of authority
% PACT (Probe, Alert, Challenge, Take Action)
% https://mobile.twitter.com/GeorgetownABLE/status/1408498438203969541