\subsubsection*{Organizational chart as a Guide and a Lie}

An \href{https://en.wikipedia.org/wiki/Organizational_chart}{organizational chart} (hereafter ``org chart'') identifies formal roles and the formal relations among roles. An org chart is at best a snapshot in time, and more often aspirational than descriptive. In spite of possible deficiencies, an org chart helps outsiders and newcomers understand the scope of responsibilities and interactions.\footnote{The organization chart hasn't always existed. The \href{https://en.wikipedia.org/wiki/George_Holt_Henshaw\#First_organization_chart}{first known org chart} was created in the 1850s.}

Org charts are a lie in the sense that undocumented relationships can matter more than the official roles. Org charts fail to capture the informal roles and network of relations that facilitate progress in any organization. 

Org charts foster a second separate lie by creating sense power dynamics based on visual orientation. For more on this issue see section~\ref{org-chart-orientation}.
