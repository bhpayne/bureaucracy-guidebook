\subsection{Static and Dynamic Process\label{sec:static-dynamic_processes}}

% https://graphthinking.blogspot.com/2017/04/static-versus-dynamic-processes.html

Change within an organization is to be expected since the external environment the organization exists in is not static. Internal change is driven by staff turnover (see \S\ref{sec:turnover}) and changes of technology. With that premise, why are static processes that are not robust to change created in the first place? Answer: so policy creators can point to an effort and avoid looking neglectful.

Creating robust processes that are dynamic takes more effort to create. First, the underlying process must be documented. What is expected to happen? Who are the stakeholders? This logic is fragile to change. Second, document assumptions used in the process. If the assumptions are invalidated, then the process is broken and needs to be discarded or at least revised. 

The problem is that even when the process is documented and assumptions enumerated, they may not be checked to see if revision is necessary. Measurements need to be periodically taken to see if the assumptions are still applicable. Another approach is to use \href{https://en.wikipedia.org/wiki/Sunset_provision}{sunset provisions} -- there is an intent of forcing renewal. 

A more quantitative approach is to tie a process to a cost/benefit model. The benefit to implementing a process comes at some cost. If the assumptions of the process can be tied to a cost/benefit model, then we can determine whether the process is worth implementing. Periodic measurements are needed to update the cost/benefit model and determine whether the process is effective.

The steps for creating a robust process are
\begin{enumerate}
    \item Document the process that is fragile.
    \item Enumerate assumptions used in the process. 
    \item Relate the assumptions to a cost-benefit model
    \item Determine the measurable parameters of the cost-benefit model. 
    \item Collect recurring measurements to verify the assumptions. 
    \item If the assumptions are broken, revise the process. 
\end{enumerate}
A robust process is just a fragile process with a feedback loop informed by ongoing measurements. 