\section{Promotion\label{sec:promotion}}

Bureaucrats get promoted for multiple reasons: due to their technical competence, their managerial capability, or to get them out of the way.

\ \\

Promotion typically focuses on the successes of the individual. This deemphasizes teamwork and lessons learned from failures. Lowering the value of learning from failures increases risk aversion. The risk tolerance of a bureaucratic organization is driven in part by tensions in the promotion processes: an interest in success (more success is better) and avoidance of failure. This imbalance can lead seeking ownership of efforts that are likely to succeed, seeking easier successes, and afterwards stealing credit for outcomes created by other people. 

\ \\

\href{https://en.wikipedia.org/wiki/Peter_principle}{Peter principle}\marginpar{[Tag] Folk wisdom}: 
\index{folk wisdom!\href{https://en.wikipedia.org/wiki/Peter_principle}{Peter principle}}
``people in a hierarchy tend to rise to `a level of respective incompetence': employees are promoted based on their success in previous jobs until they reach a level at which they are no longer competent, as skills in one job do not necessarily translate to another.''

% https://graphthinking.blogspot.com/2021/03/a-tension-between-working-at-discount.html
This might manifest as recurring failure. More likely, your incompetence will blind you to your own failures.

To avoid falling victim to the peter principal, do not promote unless the employee has a track record of performing at the next promotion level. The downside is that the employee is working at a discounted rate since they are providing value beyond what is expected of their grade.


\href{https://en.wikipedia.org/wiki/Dilbert_principle}{Dilbert principle}\marginpar{[Tag] Folk wisdom}:
\index{folk wisdom!\href{https://en.wikipedia.org/wiki/Dilbert_principle}{Dilbert principle}}
``systematically promote incompetent employees to management to get them out of the workflow.''\\
and\\
\href{https://en.wikipedia.org/wiki/Putt\%27s_Law_and_the_Successful_Technocrat}{Putt's Law}\marginpar{[Tag] Folk wisdom}.
\index{folk wisdom!\href{https://en.wikipedia.org/wiki/Putt\%27s_Law_and_the_Successful_Technocrat}{Putt's Law}}





% https://graphthinking.blogspot.com/2021/04/notes-from-peter-principle-by-peter-and.html
Peter principal \cite{1970_Peter} has an element of truth, especially initially in a person's assumption of a new role. However, a person can learn their role and become competent. Also, the Peter principles relies on the simplification of a single dimension of intelligence. 
See \href{https://en.wikipedia.org/wiki/Theory_of_multiple_intelligences}{Gardner's theory of multiple intelligences}.


Promotion is a critical aspect of designing incentives for behavior. Promotion is central because there are a variety of motives -- for money, for status, for authority. 

Promotion of the individual (rather than team) results in hero culture.

What is preventing innovation is lack of risk taking. Actual risk means potential for failure. fear of failure because culture avoids failure due to concerns of waste. Also, anyone who fails can use this argument, but that isn't the same as ``fail fast'' because what's critical is learning from failure and sharing that insight gained so other's don't need to repeat. 

There are two options for learning: your own mistakes, or the mistakes of others. Formal education is about the latter.
