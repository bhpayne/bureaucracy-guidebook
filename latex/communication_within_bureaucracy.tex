\section{Communication within a Bureaucracy}

\subsection{Communication is Critical for Bureaucracy}
\gls{bureaucracy} is a system of distributed knowledge and distributed decision making. In that context, the importance of effective communication cannot be overstated. 
\href{https://en.wikipedia.org/wiki/Metcalfe\%27s_law}{Metcalfe's law} says the value of an organization is proportional to the square of the number of people interacting in the organization. Broadening your network of collaborations enables your effectiveness and that of your fellow bureaucrats. 

Your sense of being saturated with meetings, emails, phone calls, and other coordination may stem from ineffective communication. 
Communication that is incorrect, imprecise, redundant, and insufficient must be improved. 
% https://graphthinking.blogspot.com/2016/04/impact-of-communication-on-negotiation.html
Poor communication yields poor negotiation. Then the interaction defaults to an emotional struggle of will power. 

%*************************************
When a challenge that spans multiple bureaucrats (and may span teams) is recognized in a bureaucracy, a typical response is to hold more meetings (ok for decision making) and send more emails (ok for sharing information). Both these channels are ephemeral. Share information using channels that are discoverable and durable. 

%*************************************

In this book I don't address paper memos versus email versus phone calls versus video chat versus Skype versus Slack. 
The relevant attributes are agnostic to specific technologies and implementations: synchronous versus asynchronous; discoverable or not discoverable; searchable or not searchable. 
%In that context, there might be some interesting bureaucratic specific things to think about.

%*************************************

Communication within a bureaucratic organization occurs on a gradient of how official the information is. Examples of the gradient ordered from unofficial to official:
\begin{itemize}
    \item I just made this up and I have no authority
    \item Based on my experience, ...
    \item Based on my training, ...
    \item I heard from someone else (folklore from a peer)
    \item I heard from someone in a position of authority
    \item Referencing a written document
    \item Referencing a written document that looks official
    \item Referencing a written document that looks official with the signature of a position of authority
\end{itemize}
These indicators convey the relative importance of information for the bureaucrat. 

%*************************************
\ \\

Information within a bureaucracy shapes the relationships of bureaucrats. That spans from not sharing information (which can harm or protect relationships) to being transparent to being vulnerable.  
Being transparent conveys ``here's what is happening." Being vulnerable expands that to ``here's why that's happening or what might happen."

\ \\