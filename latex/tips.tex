\subsection{Tips for working in an organization}

superpowers in a bureaucracy are
\begin{itemize}
    \item multi-tasking (task switching when I encounter a wait)
    \item quickly replying to incoming requests
    \item not dropping tasks or requests
\end{itemize}

Bureaucracy induces emotional response in participants because things don't work they way you want. This can lead emotionally to frustration and then apathy. Understanding how things operate in a bureaucracy decreases the anger.

Another emotional response to bureaucracy is a sense of powerlessness. 
\begin{quote}
Some third person decides your fate: this is the whole essence of bureaucracy.\footnote{``The Workers' Opposition'' by Alexandra Kollontai, 1921}
% https://alphahistory.com/russianrevolution/kollontai-on-soviet-bureaucracy-1921/
\end{quote}

Wandering the maze of bureaucratic processes as a subject.

Be prepared with a backlog of ideas (in writing) if someone shows up with resources.

When you are asked to take on additional work, avoid responding with ``that's not my job." If the request is misguided and your perception is that is really is the responsibility of another person, ask if the requester is aware of that other person's responsibilities. If you are to really take on the work, get guidance on re-prioritizing and make sure the request is documented in writing. 

Presence creates priority.

If there's something you want to accomplish, strive for influence without authority instead of working to gain control over resources (e.g., through promotion). Avoid the following: ``In this organization X is important to me, but I can't do X right now because I don't have enough power in the org. So I'll get promoted and then do X."


Bad: no meeting agenda\\
Good: agenda\\
Better: agenda share with other participants

