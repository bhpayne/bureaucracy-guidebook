\section{Bureaucrat's View of their Organization\label{sec:alternative_views_from_within}}

Bureaucracy as I have defined it in \S\ref{sec:define_bureaucracy} is not the only way that bureaucrats perceive their environment. Bureaucrats typically focus on themselves, their work, their subjects, their coworkers, their superiors, and their subordinates. The motives of an individual bureaucrat for an activity varies (see \S\ref{sec:motivations}). The variance of motivations is even more complicated when a request costs additional work and there's no deadline and there's no reward. What is your incentive? Is it emotional approval? Relationship building? Social approval?

\ \\

The perspectives below are archetypal for bureaucrats who don't consider bureaucracy as defined in \S\ref{sec:define_bureaucracy}. In practice an individual's perspective might be a mixture of these views.

\ \\
\textbf{As a bureaucrat, what matters is what I can accomplish with my skills and the resources I have access to.} \\
\textit{Assessment}: This person is task oriented. Results are what matters. The intricacies of bureaucracy are a distraction to getting the work done. 
% https://graphthinking.blogspot.com/2015/05/a-method-for-herding-cats.html
Communication  for the purpose of coordination is a distraction from work of the individual. 
This bureaucrat may have the perception that if they initiate formation of a community, they will be blamed when things go wrong.
The emotional reward for this person is accomplishment of the task. This person is likely to say to their manager, ``Tell me what I need to do to be successful" rather than identify collaborations.

\ \\
\textbf{As a bureaucrat, what matters is how I feel.} \\
\textit{Assessment}: Your feelings are real. They have consequence, in that your emotions impact motivation and enthusiasm. However, a feelings-centric perspective may not be productive for you or your team or the organization. Being effective means compromise and some people may not get everything they wanted. Balancing those competing needs is challenging.

\ \\ 
\textbf{What matters is how others feel.}\\
\textit{Assessment}: Depending on the emotional state of those around you is unhealthy and can be unproductive. Working for the happiness or satisfaction of other people is risky -- they may not know what's best, or they may not have your interests in mind.

\ \\
\textbf{What matters are my immediate coworkers.}\\
This perspective can be positive (I collaborate with those around me) or negative (I am in competition with those around me).
In this scenario everything beyond the local scope is personified or ignored.  \\
\textit{Assessment}: Your relationships do matter. However, they are not all that matters. Missing from this view is the ability to explain what is happening outside the immediately observable realm. 

\ \\

Just because you are a bureaucrat doesn't mean you have a well-informed understanding of bureaucracy. Regardless of their perspective, a bureaucrat certainly experiences the difficulties of operating within an organization. A bureaucrat can rationalize to themselves why things don't work in their organization with stories like
\begin{itemize}
\item other people are lazy and don't want to work.
\item other people are inexperienced.
\item other people don't care.
\end{itemize}
There certainly are lazy people, inexperienced people, and people who don't care in any given organization. Those are not unique to bureaucracy and do not explain bureaucracy.

The bureaucrat who uses the explanations like laziness, lack of experience, and lack of care applies them to people he or she hasn't directly interacted with.  The person using these explanations may not realize other people could use those same stories. 