\section{Intellectual Empathy}

As a bureaucrat, you will work with people who are smarter on a specific topic than you are. That superior knowledge might come from experience or from formal education. 

The person with superior knowledge may be able to convey their knowledge to you, or they may struggle to. That struggle could be due to an inability to break down complex insights, or they don't recognize the mismatch between what they know and what you understand. 
This same scenario can be reversed -- you might be more knowledgeable on a given topic than some other bureaucrat you work with. In either situation, both parties have responsibilities of intellectual empathy. The more knowledgeable person needs to understand what they can build on when talking with the less knowledgeable person, and the less knowledgeable person has to be clear about what is making sense and what does not. 

I often start conversations with people who are new to me by asking about their professional background -- their education, past work experience, and other aspects that would allow me to build on their existing knowledge. This allows me to tailor the conversation to the gap of knowledge that exists when we are discussing a challenge. 

During a conversation you can monitor body language for reactions like a squint, a head tilt, an eyebrow raise or an eyebrow furrowed. Those are common reactions when information doesn't conform to what the listener was expecting. 

A more direct method is to ask about the other person's knowledge about the topic. The phrasing ``Do you know about (name of topic)?'' is less effective than ``What do you know about (name of topic)?'' Respondents might claim experience or knowledge of a topic, but probing for the difference between ``I've heard those words before'' versus ``I was part of a team that implemented (topic)'' versus ``I came up with the original idea for (topic)'' is helpful. 

Sometimes the person with less knowledge on a topic feels uncomfortable with indicating their confusion, or sometimes the person doesn't realize they are missing the main point. The more knowledgeable person can check in with the listener to confirm. ``What did you get from what I just talked about?'' is a better confirmation than ``You understood what I meant, right?'' The person with less knowledge can proactively check their knowledge by providing a read-back: ``Here's what I understood you to mean: ...'' These check-ins, regardless of who initiates, should be relatively frequent when exploring complex topics. 

Admitting ignorance can be intimidating and may feel like you risk losing the other person's respect. A more gentle way of saying ``I'm lost'' is ``I was following up until the point where you described ...'' Intellectual empathy is the responsibility of both parties in a conversation. 