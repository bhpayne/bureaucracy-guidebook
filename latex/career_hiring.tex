In \S\ref{sec:hiring} I narrate from both the perspective of the person hiring and the person being hired. In \S\ref{sec:promotion} I describe the incentives of the person promoting a bureaucrat and the person seeking promotion. In \S\ref{sec:professional-training} I explain the relevance of training from the perspective of the employer and the employee. My intent is to help both sides build empathy with the other person's experience. 


\subsection{Hiring into a Bureaucracy\label{sec:hiring}}


Hiring shapes the culture of an organization and determines what is feasible, both by the skills of those hired and how well new hires integrate with the existing organization. 

Hiring is an expensive process in terms of money, time, emotional investment of candidates, and the burden of reviewing applicants. 
% SO WHAT? What's the consequence?
As a candidate, recognize the risk the organization is taking. 
As a reviewer of candidates, this is an investment in the future of the organization. Even if your bureaucratic role does not require an exclusive focus on hiring, there is value in studying this domain. 

When hiring into an organization there is a selection bias in the people who join a bureaucracy, and there is a selection bias who stays in bureaucracy. 
% SO WHAT? What's the consequence?

% https://leadership.garden/onboarding-engineers/
% https://news.ycombinator.com/item?id=30810786

Regardless of the specifics of the job, there are specific attributes that make a candidate more likely to be successful in a bureaucracy. In addition to role-specific skills, hire for \href{https://en.wikipedia.org/wiki/Metacognition}{metacognition}, social skills, and intrinsic motivation.

% https://graphthinking.blogspot.com/2021/04/screening-for-metacognition-in-job.html

% https://graphthinking.blogspot.com/2021/07/screening-for-intellectual-empathy-in.html

% https://graphthinking.blogspot.com/2021/04/questions-to-ask-interviewer-when.html

