\subsection{Measuring Bureaucratic Maturity}
% https://graphthinking.blogspot.com/2021/07/three-measures-of-bureaucratic-maturity.html

bureaucratic maturity can be broken into three behaviors: 

The first and most wide spread behavior is to observe a problem and then complain about the situation. 

The second, less common behavior, is to observe a problem and recognize the situation as an opportunity. 

The third behavior is to observe a problem and then nudge the situation towards a vision. 



An individual bureaucrat can exhibit one or more of these behaviors. That is, being capable of the third behavior does not mean an inability to complain. 

There is not a specific amount of experience within the organization needed to arrive at any of these three paradigms. A holistic understanding of the system certainly helps.

The "vision" of the third behavior could be in the form of a long-term (temporally distant) outcome, or the vision could be of immediate multi-party cooperation. 