\subsection{Decision making in a bureaucracy}

There isn't one answer, much less a correct answer. If there were, there wouldn't be a need for a decision. Even afterwards a decision can be difficult to evaluate for correctness.

In bureaucratic processes there is rarely a formal assessment of options. 
Decisions are rarely recorded. 

Decision making by bureaucrats can be informal or formal, consensus-based or solo. 

Most decisions made by bureaucrats do not have hard deadlines. Instead, there are trade-offs in timing. Sooner is better, but delaying allows for more information gathering for a better informed decision.


If a bureaucrat is going to rely on expert consultation, the decision maker needs to be confident the expert is not of straying outside their area of expertise. For example, I don't rely on a botanist with many published papers to tell me how to change the oil in my car. 

When getting input for a decision, is the expert providing a factual summary, a predictive assessment, or a value judgement? 

There are many ways to gather evidence. 
Role of measurement and modeling. 
Form an opinion, look for evidence to back the outcome.
Instead of measurement, most people rely on history (if they are aware of it), or what is best for their career, or how to accumulate more power, or what someone else says to do.  