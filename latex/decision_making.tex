\subsection{Decision making in a bureaucracy}

There isn't one answer, much less a correct answer. If there were, there wouldn't be a need for a decision. Even afterwards a decision can be difficult to evaluate for correctness.

Decisions are rarely recorded, an there is no formal assessment of options. 

Decision making can be informal or formal, consensus-based or solo. 

Most decisions do not have hard deadlines. Instead, there are trade-offs. Sooner is better, but delaying allows for more information gathering for a better informed decision.


Relying on expert consultation? Need to be wary of straying outside the area of expertise. I don't rely on a botanist with many published papers to tell me how to change the oil in my car. 

Is the expert providing a factual summary, a predictive assessment, or a value judgement? 

Role of measurement, modeling. 
Form an opinion, look for evidence to back the outcome.
Instead of measurement, most people rely on history (if they are aware of it), or what is best for their career, or how to accumulate more power, or what someone else says to do.  