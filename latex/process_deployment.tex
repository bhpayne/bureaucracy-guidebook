\subsection{Deployment of Processes, aka policy update}


Similar to product deployment in that similar dilemmas are faced

Deployment of processes and products need to account for 
normal users, Power users, malicious users, and edge cases

\href{https://en.m.wikipedia.org/wiki/The_Innovator's_Dilemma}{Innovator's Dilemma} applies within bureaucracies to policies and processes

\href{https://en.wikipedia.org/wiki/Diffusion_of_innovations}{Diffusion of Innovation}

\subsubsection{external to the org}


\subsubsection{Internal product development and deployment\label{sec:internal_product}}

Teams in a bureaucratic organization 
\begin{itemize}
    \item consume from outside-the-organization
    \item consume from within-the-organization
    \item produce to outside-the-organization
    \item produce to within-the-organization
\end{itemize}

This section focuses on the relation between teams that produce to within-the-organization and teams that consume from within-the-organization. Tools or products that are created internally and consumed internally.

Feedback mechanisms and incentives in a non-profit monopoly. The claim of success that a team created a product that met all design requirements on-time may have no actual benefit to users. Or maybe a product that benefited internal customers was created and there were happy users, but the originating team has to quickly move onto the next project to create another success and thus has no attention to on-going support. Determining the metric of success is tricky. do you want to aim for highest average happiness of stakeholders, or are some more important?

captive users who have little leverage 



