\subsection{Characterizing Meetings}

Characterizing meetings is critical to distinguishing which norms are applicable, and what people expect from the different formats. 

types of meetings: internal meetings, customer meetings, conferences 

% https://graphthinking.blogspot.com/2019/12/what-is-purpose-of-this-meeting.html
Many purpose to a meeting
\begin{itemize}
    \item To gather input from attendees
    \item To make a pronouncement to attendees
    \item To educate
    \item To brainstorm ideas
    \item To make progress towards an objective
\end{itemize}
When the purpose is not explicitly stated, confusion arises.

When multiple purposes occur in one meeting and the transition is not explicitly stated, confusion arises.
The reason for this confusion is that the assumptions and expectations and norms of each purpose are different. When the attendees don't know the purpose or the purpose shifts, the behaviors and roles are in question.

% https://graphthinking.blogspot.com/2014/12/how-to-understand-meetings-at-work.html
Level of formality, start time (early | on time | late), 
end time (early | on time | late), utility, 
duration, number of attendees, number of speakers, number of participants.


What is the purpose of a meeting?

meetings involve people, either known or strangers
meetings involve information, either relevant or irrelevant. Relevant information is either new or related to previous work
meetings either have a leader or no leader (brainstorming). If there's a leader, the leader may be disseminating info to participants, or gathering information from attendees