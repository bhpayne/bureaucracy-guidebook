\subsection{How to be a Successful Conference Attendee}

% https://graphthinking.blogspot.com/2012/02/how-to-succeed-as-attendee-at.html

\textbf{Before the conference}\\

Determine the date of the conference. Does it conflict with other events in your schedule?

How will attending the conference benefit your progress? Social interaction, presenting your ideas.

Submit your abstract for review prior to the deadline.

Apply for travel grant funding as needed. Look to the sponsoring organization, conference website, and your own host institution.

Find housing nearby the conference location. Find a roommate.

Book a flight/transportation. Plan to arrive at least a day before you are scheduled to speak. (Speaking on the day you arrive is tiring.) Flights on Monday and Friday are often more expensive than mid week. Cheapest flights are released on Tuesday. [link]

Coordinate someone to cover for you while you are gone.

Find out who else is attending the conference. Which talks look interesting. If there are people you know, let them know you are coming and would like to see them.

Start stalking attendee profiles. If it is a small conference/summer school, get a list of attendees and learn their names/interests.

Bring printed papers, CV, business cards. Distribute these at the career fair, your talk. Bring an night shade and ear plugs. Your roommate may snore.

\textbf{How to determine which sessions to attend}\\

For very large conferences, the scope is broader so there will be a lot of content outside your field. This means if you go to short talks, they will be hard to understand since they are very focused. Therefore attend the invited and plenary talks (usually at the beginning of the week) even if they are outside your field. They will be at a more accessible level, helping to determine which subfields may be of interest.

Start socializing on the plane, in the airport. This is warmup for the conference.

\textbf{Upon arrival}\\

Register. Get map, current conference schedule.

Socialize as much as possible. This is the point of a conference. Talk to people in line, before and after talks, at meals, at the hotel. Advantages over a regular stranger are that (1) there is a common topic to discuss (other than the weather), (2) they are probably also interested in that common topic. Also discuss meta-topics -- how the talk went visually/audio, how the conference is being run. Both parties are probably out of their home environment/city, so that is another commonality.


Keep all receipts for purchases. Keep your airline boarding passes.

Write down interesting ideas that are triggered by conversations outside your specialty.


\textbf{After your return}\\

Send emails to people you spoke with/got business cards. "Thanks for talking with me about X at the Y conference last week."
For recruiters, "Thanks for talking with me about positions X at the Y conference last week. Attached is a copy of my curriculum vitae. Please contact me to schedule an interview."