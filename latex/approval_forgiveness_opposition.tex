\subsection{Tactic: Approval, Forgiveness, Opposition}
% https://graphthinking.blogspot.com/2017/10/flipping-approval-mentatlity.html

A common bureaucratic task is consensus regarding action or expending resources. There are distinct options about how to get that consensus:
\begin{itemize}
    \item Seek approval before taking action. This approach incurs both providing justification and waiting.
    \item Ask forgiveness after taking action. Often viewed as being in contrast to seeking approval. Less delay, and usually works if things go well or if no one notices. 
    \item Notification of Intent with deadline for response. The window for response should be sufficient to actually allow feedback. 
    \item Solicit opposition before taking action. This is a different framing from approval or forgiveness. It decreases the risk the approver has to take on.
\end{itemize}
The best way to proceed depends on the personalities of the people involved in building consensus and their relationships. 

Most organization default to an approval-based  processes. Each new idea needs to be signed off as approved by a sequential list of bureaucrats. The sequence (not concurrent) process may be known in advance, or it may be ad hoc if the request is novel.

Relying on approval is harmful to innovation because sign-off by each bureaucrat is interpeted as ``I am 100\% in agreement with this.'' Each stakeholder has to bless innovation and tie their reputation to the outcome.

The ``I won't stop this'' (soliciting opposition) is a more useful paradigm. With the consensus process language changed to "I won't stop this," then the bureaucrat can avoid taking responsibility for the idea and therefore are not tying their reputation to the result.

% https://news.ycombinator.com/item?id=15407757