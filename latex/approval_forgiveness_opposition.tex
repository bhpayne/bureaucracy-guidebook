\subsection{Tactic: Approval, Forgiveness, Opposition}
% https://graphthinking.blogspot.com/2017/10/flipping-approval-mentatlity.html

A common task is consensus regarding action or expending resources. There are distinct options about how to get that consensus:
\begin{itemize}
    \item Seek approval. Incurs both providing justification and waiting.
    \item Ask forgiveness. Often viewed as being in contrast to seeking approval. 
    \item Solicit opposition. 
\end{itemize}
The best way to proceed depends on the personalities of the people involved in building consensus and their relationships. 

Most organization default to approval processes. Each new idea needs to be signed off as approved by a sequential list of bureaucrats. The sequence (not concurrent) process may be known in advance, or it may be ad hoc if the request is novel.

Relying on approval is harmful to innovation because sign-off by each bureaucrat is interpeted as ``I am 100\% in agreement with this.'' Each stakeholder has to bless innovation and tie their reputation to the outcome.

The ``I won't stop this'' is a more useful paradigm. With the consensus process language changed to "I won't stop this," then the bureaucrat can avoid taking responsibility for the idea and therefore are not tying their reputation to the result.

% https://news.ycombinator.com/item?id=15407757