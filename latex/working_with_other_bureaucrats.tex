\section{Working with other Bureaucrats\label{sec:working-with-other-bureaucrats}}


Loss of independence within the org (deindividuation) is either sublimation or extension 

% https://graphthinking.blogspot.com/2016/07/if-everyone-is-doing-right-then-why-are.html
If everyone participating in a process thinks they are doing the right thing, then why are there suboptimal outcomes?
\begin{itemize}
    \item Each participant may have limited view of other parts of the process
    \item Each participant typically has limited scope of responsibility
    \item Each participant typically has limited scope of authority
    \item participants typically have insufficient time, resources, and expertise
    \item participants may have a common objective, but different methods for addressing the challenges
\end{itemize}


% https://graphthinking.blogspot.com/2017/05/presence-with-attention-time-alertness.html
When you engage with fellow bureaucrats, what you are providing is your time, attention, and alertness. To illustrate the relevance of those aspects, consider the following. I can talk to you for 15 minutes of time, but if your attention is directed towards your phone, then you're not fully engaged with me. If you are paying attention during the time we have together but you haven't slept for 36 hours, then your alertness may not be 100\%. 

\ \\

Why would you need to interact with other bureaucrats?
\begin{itemize}
    \item Delegation of tasks
    \item Asking for help
    \item Seeking input
    \item Offering to help
    \item Offering input
\end{itemize}

Understand what someone else is priorities are and why those are priorities

These all directly impact your reputation; see \S~\ref{sec:reputation}.


Enumerate tropes to figure how to respond

% https://graphthinking.blogspot.com/2021/10/why-i-dont-like-being-in-management-role.html
Solo work may be more emotionally rewarding due to fewer external constraints, but the cost is complexity and scope being limited to the skills of the individual. 

Working with others allows you to occasionally accomplish complex results beyond your own skills or your own bandwidth in spite of collaborators not being under your control. How? Through persuasion. 

The challenge of collaboration is to multiply productivity rather than merely sum the output of a set of individuals. 

inside an organization, cooperation/coordination is not held together by internal contracts or even service level agreements. What holds the organization together? Force of will of participants. 

two networks: formal and informal