\section{How to be an Effective Bureaucrat\label{sec:effective_bureaucrat}}

So far I've described dilemmas in \S\ref{sec:dilemma_trilemma} and unavoidable hazards in \S\ref{sec:unavoidable_hazards}. Given this nuanced view of the complexity of bureaucracy, how can you leverage that insight?

Baseline assumptions are that you are a good person, you are technically skilled at your technical duties, and you are a good project manager. 


Superpowers of a bureaucrat in a bureaucratic organization that facilitate cooperation and progress:

\begin{itemize}
\item You strive for and demonstrate transparency. You share information with stakeholders.
\item You seek information from stakeholders.
\item You apply consistent processes (rather than being reactionary and applying ad hoc responses).
\item You hold others (and yourself) accountable for their actions.
\item You account for varying incentives and reference experiences.
    \item You communicate more effectively than anyone around you.
    \item Your effectively multi-task, or, more accurately, switch tasks. The switch among tasks is triggered when the current task encounters an externally-generated pause. You can \href{https://en.wikipedia.org/wiki/Pipeline_(computing)#Concept_and_motivation}{pipeline your activities}.
    \item You are prepared with a backlog of ideas (in writing) if someone shows up with resources.
    % https://graphthinking.blogspot.com/2016/12/life-lessons-i-learned-from-experience.html
    \item If you are dependent on someone else getting something done to enable your progress, you can demonstrate priority by physically show up -- \underline{presence creates priority}. Being physically at a person's desk motivates that person to respond better than calling them or emailing them. Showing up where someone works and talking with them conveys how much priority you place on the actions of the person you're talking with.
    \item You reply quickly to incoming requests. This allows other people's tasks to either be resolved or have a clear response about when progress can be expected. 
    \item You have intellectual empathy -- theory of mind for thinking (whereas empathy refers to emotions). How to grow your intellectual empathy: shadow peers and bosses
    \item You have process empathy. Observe deviations and exceptions that cause processes to come into existence. 
    \item You avoid unintentionally dropping tasks or requests. This requires capturing incoming requests and then providing a response about prioritization and status. This matters because other participants in the organization should regard you as reliable in a positive sense. 
    \item You focus on value delivery in relationships that exceed the scope of your formal role.
\item You have altered your job description to fit the growth you're seeking.
\item You are willing to engage on a personal level and know someone outside their professional role.
\item Each of your tasks have a customer, a deadline, and a deliverable.
\item You occasionally ponder and discuss with other people introspective questions like
\begin{itemize}
    \item How can I be successful?
    \item What are the ways I could fail?
    \item How the organization is regarded as successful?
    \item What are the ways the organization can fail?
\end{itemize}
% https://graphthinking.blogspot.com/2017/07/questions-to-ask-mentors.html
\item You find mentors and ask them questions like
\begin{itemize}
    \item What do you like most about your career? 
    \item Given the chance, what would you do differently?
    \item How do you manage work/life balance?
    \item What's the big challenge for our industry in the next two years?
    \item How would you tackle that challenge?
    \item What advice do you have for a young person starting in this industry? Mistakes to avoid?  How to be successful?
    \item What books would you recommend reading?
\end{itemize}
\end{itemize}


A bureaucrat can accomplish more as part of an organization than by working alone. Being a member of an organization means the bureaucrat's identity is subsumed into service for the organization they are part of~\footnote{\href{https://en.wikipedia.org/wiki/Deindividuation}{Deindividuation}}. At the same time, bureaucracy enables the bureaucrat to amplify their presence by being part of a larger organization.  Sometimes the cost of being part of the organization exceeds the force multiplier of working together. 




% https://graphthinking.blogspot.com/2019/05/definition-of-progress.html
Measuring your own personal growth in a bureaucracy is difficult due to the lack of feedback loops. One approach is to measure your capabilities with respect to a specific task. If you can complete the task in less time and with fewer resources and with less effort, that's progress. If you can now complete a task that you previously wanted to but weren't able to, that's progress.



Bureaucracy induces emotional response in participants because things don't work they way each person wants. This can lead emotionally to frustration and then apathy. Understanding how things operate in a bureaucracy can help decrease the anger.

% Wandering the maze of bureaucratic processes as a subject.

Another emotional response to bureaucracy is a sense of powerlessness. 
\begin{quote}
Some third person decides your fate: this is the whole essence of bureaucracy.\footnote{``The Workers' Opposition'' by Alexandra Kollontai, 1921}
% https://alphahistory.com/russianrevolution/kollontai-on-soviet-bureaucracy-1921/
\end{quote}
That sense of powerlessness applies both to bureaucrats and to subjects of bureaucracy. 

The sense of powerlessness is somewhat valid, in that you are as a bureaucrat giving up some power compared to your ability to act individually. That is the trade for working with others

\ \\

A process feels bureaucratic when the subject is exposed to more than one step to get something done that if one person were doing it would be simpler. Consolidation from the subject's perspective decreases the apparent bureaucracy. The barrier to implementation is the necessary coordination among different stakeholders who receive no benefit from the coordination. Externalizing the coordination to the subject is what causes the sense of bureaucracy. 


\ \\

Don't rely on one person or one idea for your success. On the other end of the spectrum, don't spread yourself too thin. 

\ \\

What to do when you get stuck?\\
look up stream, look downstream, look to peers

\ \\

When you are asked to take on additional work, avoid responding with ``that's not my job." If the request is misguided and your perception is that is really is the responsibility of another person, ask if the requester is aware of that other person's responsibilities. If you are to really take on the work, get guidance on re-prioritizing and make sure the request is documented in writing. 

\ \\

If there's something you want to accomplish, strive for influence without authority instead of working to gain control over resources (e.g., through promotion). Avoid the following: ``In this organization X is important to me, but I can't do X right now because I don't have enough power in the org. So I'll get promoted and then do X."

\ \\



\textit{DO}: \textbf{Learn the perspectives of those around you.}\\
The relevance of knowing the paradoxes including dilemmas and unavoidable hazards, is that you should talk explicitly to your fellow bureaucrats about these specific topics in conversation. Not that the goal is to find consensus or agreement. But to find what the other person is thinking so that you can account for their processes


\textit{DO}: \textbf{Account for holistic view}\\
The specific circumstances of the challenges you face as a bureaucrat depend on the individual people involved, what the purpose of the bureaucracy is, what technology is available for implementation of bureaucracy, and the resources (staffing, money, time). 

\textit{DO}: \textbf{Learn the history of the problem}\\
This goes beyond Chesterton's fence, which focuses about why the current approach is in placed. Learning the history of a problem means what has been tried before and failed? How did the previous iterations evolve into the current situation? Was the cause personalities, insufficient resources, inadequate technology? What's changed that enables this approach to be better? What do you know that prior attempts didn't?

\textit{DO}: \textbf{Concurrently work on 3 remedies}\\


\textit{DO}: \textbf{Minimize \href{https://en.wikipedia.org/wiki/Externality}{external costs}}\\
A solution that externalizes costs harms the greater organization and creates bureaucratic debt.


\textit{DO}: \textbf{Exploit the flexibility of rules for the benefit of all parties.}\\
% https://graphthinking.blogspot.com/2019/07/winning-game.html
change the rules of the game and the objectives of the game such that every participant wins.

\textit{DO}: \textbf{Align selfish interests with social interests.}\\
See also nudges. 
(Page 66 of \cite{2012_Schneier})

\textit{DO}: \textbf{Find ways to rephrase negative complaints.}\\
% https://graphthinking.blogspot.com/2019/07/how-to-rephrase-negative-observations.html
Negative observation: "Logging into my computer takes a long time."\\
Positive statement and explanation of impact: "If the latency for logging into my computer were lower, I could make more progress on X."


Negative observation: "The service team I need support from doesn't offer a ticketing queue."\\
Positive statement: "If the service team I need support from offered a ticketing queue, I would be able to track the work done on my behalf."

\textit{DO}: \textbf{Share lessons learned}\\