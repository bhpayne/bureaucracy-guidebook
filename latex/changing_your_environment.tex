\subsection{Changing your Environment}

% https://quoteinvestigator.com/2017/05/23/culture-eats/
%\begin{quote}
%    Culture Eats Strategy for breakfast
%\end{quote}
%\href{https://en.wikipedia.org/wiki/Peter_Drucker}{Peter Drucker}

This section addresses changing the bureaucratic organization you are a member of. Change can be with respect to processes (described in section~\ref{sec:change_a_process}) or changing the \gls{culture} of the organization.

Culture is the set of norms that determine acceptable behavior for members of the organization. Norms come from the expectations of your fellow bureaucrats. Your coworkers can model desired behavior and inflict punishments for deviations from norms. The culture of the organization is practiced by members. 

% https://graphthinking.blogspot.com/2021/01/why-active-shaping-of-culture-is.html
The expectations of culture should be stated explicitly. If the explicitly stated norms are in conflict with what is experienced by the organization's members the hypocrisy decimates trust. 

Besides modeling desired norms in the organization, there are a few additional tactics you can take. 
\marginpar{[Tag] Actionable Advice}
Reinforce positive examples of desired behavior with public praise and meaningful rewards. Punish examples of undesirable norms. 

Altering cultural norms of a bureaucratic organization means altering member behavior, adjusting incentives to re-enforce the desired behavior, punish undesirable behavior, and provide resources to make the intended outcomes easier. 

Although aspects of culture may be problematic, thinking of cultural change as problem solving is not a useful framing. Bureaucrats don't solve problems, they address challenges. The word ``problem" implies solution. The problem-solution framing is inaccurate in a bureaucracy since altering the situation merely introduces novel challenges. 
Even though you don't solve a challenge, you can still get a sense of emotional reward from having improved the situation. 

Prior to investing effort in change, reflect on the emotional and career impact of your effort. Will the work bring you joy, or is your emotional reward tied to the outcome? If the outcome is the only source of reward, you'll either be disappointed in failure or you will be more conservative (intending to increase the likelihood of success). The work of creating positive change can be joyful: you can form relationships and focus on learning. 

One of the challenges of change in a bureaucratic organization is figuring out what is worth changing. There are many opportunities for improvement. Talk to stakeholders and enumerate their challenges.
\marginpar{[Tag] Actionable Advice}
From the myriad stories, identify recurring themes and root causes. Typically bureaucratic challenges intertwine technical, budget, staffing, historical context, personalities of participants, and organizational politics. 

Select an aspect that you have some competence in and allows for growth. If the challenge is big (many people involved, takes a long time, cost a lot) can it be broken down into incremental challenges? Distinguish wide-scale changes from local changes over which you have more influence.

Having surveyed challenges worth addressing and picked something, 
\href{https://en.wikipedia.org/wiki/Wikipedia:Chesterton\%27s_fence}{Chesterton's fence} says
\begin{quote}
Reforms should not be made until the reasoning behind the existing state of affairs is understood.
\end{quote}
Talking to stakeholders in the survey phase may yield tidbits of history. Why have previous efforts been rejected or not taken effect? Written documentation of the history should be augmented by oral folklore. 

If your idea has a novel justification for success, identify all possible methods of progress and then prioritize tasks. If you can't do all the work, who can you collaborate with? Can you delegate to people with skills you lack, or will you have to gain new skills? 
If no one has the relevant skills, becoming the widely-acknowledged expert accrues power. 


