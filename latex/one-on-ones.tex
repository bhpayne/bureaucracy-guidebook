\subsection{One-on-one check-in meetings}

% https://graphthinking.blogspot.com/2021/05/the-agenda-for-one-on-one-meeting.html

One-on-one meeting questions for helping the manager understand the team member's status.
\begin{itemize}
    \item what are the objectives for the team?
    \item what have you been successful with since we last met?
    \item what is blocking our team's progress?
    \item what are your plans?
    \item how are you collaborating with the rest of the team?
\end{itemize}

Reflective prompts for one-on-one meetings:
\begin{itemize}
    \item If there was just one thing you could change about our organization, what would it be and why?
    \item How do you plan to train your coworkers on topics you understand and they don't?
    \item What have you learned in the past month?
    \item What are the biggest risks for the team?
    \item What's limiting your productivity?
\end{itemize}
Responding to these questions takes time (an hour) and willingness to be open. 

\ \\

The one-on-one check-in should be tailored to the phase of the employee's progression. 
\begin{itemize}
    \item new team member, either new to the team or new to the company. Here the focus of the one-on-one is to ensure a smooth on-boarding process. Does the employee have the necessary computer log-in accounts? Do they have an email account? Are they on the mailing list?\\
\textit{The duration of this phase could last between a day and two weeks.}
    \item team member is responsible for small tasks: the one-one-one is for discussions on training and sprint planning and sprint-reviews. Characterized by the team member being dependent on others for their success. In this phase the employee collaborates on tasks.
\textit{The duration of this phase could last a few months to years.}
    \item team member is responsible for large tasks (which get broken into subtasks): the one-on-one is to help the team member define their success. Activities include planning, resource allocation, assessment. Characterized by the need to coordinate with others on the team or other teams.
\textit{The duration of this phase could could be the rest of a career.}
    \item facilitating the productivity of others: rather than being task-oriented, this team member supports coworkers. 
    \item peer check-in: this one-on-one is a form of mentorship. The value of the exchange is to get a different perspective and to hold each other accountable.
\end{itemize}

TODO: How does the team member and the supervisor know when the next phase is appropriate?

TODO: What are the thresholds for change?

\ \\

https://news.ycombinator.com/item?id=30152268

\ \\

https://news.ycombinator.com/item?id=22341138
https://github.com/VGraupera/1on1-questions

