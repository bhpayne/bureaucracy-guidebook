\section{Folk Wisdom}
While most of the entries on
 \href{https://github.com/dwmkerr/hacker-laws}{https://github.com/dwmkerr/hacker-laws}
don't apply to bureaucracy, the list is useful to review. 

% https://www.timsommer.be/famous-laws-of-software-development/

Folk wisdom is an attempt to explain bureaucratic features but ends up being \gls{thought terminating}.











== referenced elsewhere in this book ==

==== decision making in the face of dilemmas ====

\href{https://en.wikipedia.org/wiki/Hick\%27s_law}{Hick's law}\marginpar{[Tag] Folk wisdom}: ``increasing the number of choices will increase the decision time logarithmically.''

\href{https://en.wikipedia.org/wiki/Hanlon\%27s_razor}{Hanlon's razor}\marginpar{[Tag] Folk wisdom}: ``never attribute to malice that which is adequately explained by stupidity.''

\href{https://en.wikipedia.org/wiki/Parkinson\%27s_law}{Parkinson's law}\marginpar{[Tag] Folk wisdom}: ``work expands so as to fill the time available for its completion.''

\href{https://en.wikipedia.org/wiki/Murphy\%27s_law}{Murphy's law}\marginpar{[Tag] Folk wisdom}: ``Anything that can go wrong will go wrong.''

\href{https://en.wikipedia.org/wiki/Law_of_triviality}{Law of Triviality}\marginpar{[Tag] Folk wisdom}: ``people within an organization commonly or typically give disproportionate weight to trivial issues.''

==== Communication ====

\href{https://en.wikipedia.org/wiki/Allen_curve}{Allen curve}\marginpar{[Tag] Folk wisdom}: ``exponential drop in frequency of communication between engineers as the distance between them increases.''

\href{https://en.wikipedia.org/wiki/Wiio\%27s_laws}{Wiio's laws}\marginpar{[Tag] Folk wisdom}: ``Communication usually fails, except by accident''

==== Promotion ====

\href{https://en.wikipedia.org/wiki/Peter_principle}{Peter principle}\marginpar{[Tag] Folk wisdom}: ``people in a hierarchy tend to rise to "a level of respective incompetence": employees are promoted based on their success in previous jobs until they reach a level at which they are no longer competent, as skills in one job do not necessarily translate to another.''

\href{https://en.wikipedia.org/wiki/Dilbert_principle}{Dilbert principle}\marginpar{[Tag] Folk wisdom}: ``systematically promote incompetent employees to management to get them out of the workflow.''\\
and\\
\href{https://en.wikipedia.org/wiki/Putt\%27s_Law_and_the_Successful_Technocrat}{Putt's Law}\marginpar{[Tag] Folk wisdom}