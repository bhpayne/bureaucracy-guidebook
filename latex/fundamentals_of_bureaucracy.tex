\section{Fundamentals of Bureaucracy\label{fundamentals_of_b}}
  
Bureaucracy is a system of distributed knowledge and distributed decision making for managing shared resources. The decision making is not purely logical; it relies on particular facets of human interaction. This section outlines the critical roles of decision making, hierarchy, and communication in bureaucracy. 
% the following sentance acknowledges there's a difference but doesn't explain why there's a difference.
My list is smaller than Max Weber's \cite{2015_Weber}\footnote{\href{https://en.wikipedia.org/wiki/Bureaucracy\#Max_Weber}{Bureaucracy: Max Weber}} because I see bureaucracy as more widespread.

Decision making (section~\ref{sec:decision-making}) is central to bureaucracy. Every other aspect of bureaucracy derives from decision making. The decision making is with respect to the subjective management of resources. Here resources refers to attention and expertise or tangible goods like air, water, and land. 

When there are multiple people present, or even when one person is trying to be self-consistent, coordination of decision making is important to the management of resources. While coordination can occur without hierarchy when there are a small number of people, typically an organization of people leads to both formal and informal hierarchies. Who gets to make which decision is managed using hierarchy; see section~\ref{sec:hierarchy_of_roles}.

Meetings and written communication help with consensus among bureaucrats, though agreement isn't necessarily the outcome.
Once a decision is made, the choice selected by a person propagates throughout the organization to achieve some level of consistency. See section~\ref{sec:meetings-for-coordination} on meetings and section~\ref{sec:written-communication} on written communication.

If you're not an academic researcher of bureaucracy, you might be thinking that you don't want to have to think about hierarchy or collaboration or coordination. Meetings and email are not your goal; you just want to do the tasks that you were trained for. Your formal education process had individual grading. You perceive your job as a bureaucrat as having pay, promotion, hiring/firing, and title that  seem focused on your individual abilities. The following sections explain why the individualist mentality is ineffective when you are part of a  system of distributed knowledge and distributed decision making. 


% there are examples where the ability of the team is measured: 
% An example of hiring a team is when a company buys another company for the purpose of talent acquistion