\subsection{Processes involving two Organizations}

When two independently-governed processes interact, there can be friction. Process friction, especially two uncoordinated organizations, results in either exceptions to the process (\S\ref{sec:exceptions-to-process}) or lying or change in process (\S\ref{sec:change_a_process}). 
Example: receipts from one process submitted for reimbursement to another process.
Submitting the receipt shifts the accountability and justification burden to the accepting bureaucrat.



% https://graphthinking.blogspot.com/2017/02/financial-motivation-in-bureaucracy.html

I have dental insurance. I visited the dentist December 2 and one of the procedures during a routine cleaning was ``bitewing x-rays." This procedure is covered by my insurance, so I was surprised when I received a bill for it from my dentist.

I called my dentist and they explained that the dental insurance had declined to pay for the procedure. I called the insurance company Jan 3 and they confirmed that the procedure was covered by my policy. 
Every time I call the insurance provider I have to provide my SSN, DoB, and Zip twice -- once to the automated system, a second time to the person I talk with. The insurance company apparently had made a mistake and said they would cover the cost of the procedure. I followed up with my dentist and explained the situation.

I called the dentist to see if they had received payment yet. They had not, so I called the dental insurance provider again Feb 6 and 8. Feb 8 the insurance company said they would process the payment within 7-10 business days. I called again Feb 20 and the claim hadn't been initiated within the insurance company. I spoke to the supervisor and she said she would personally visit the claims office within the insurance company.

From the perspective of the dentist, they are seeking money for the service they provided me.

From the perspective of the insurance company, delaying payment on a claim makes good financial sense -- the policy holder is likely to just pay the balance in order to avoid going to court with the dentist.

From my perspective, the question is whether chasing this issue makes financial sense. I think of my hourly rate as $40, so after an hour the charge of $38 would have been better to pay out of pocket. Effectively I'm devaluing my time. The emotional stress and thought-cycles expended are also relevant, though harder to quantify.
