\section{Models of Bureaucracy}

Commercial businesses have a different accountability -- money. Common across all participants within the organization, and common with external stakeholders. The goal of a company is to generate profit. Commercial businesses have people who make subjective decisions and enforce policies, but there is a common metric for feedback. The feedback mechanism is not perfect. Being a good commercial bureaucrat does not necessarily result in monetary success.

Prisons, schools, medical, government, military all consume and spend money, but money isn't the goal. When faced with a decision, choice is not guided by which will generate more profit.


Bureaucracy resists characterization. Bureaucracy is worse than emergent - the system rules can be altered or ignored by the stakeholders. \href{https://en.wikipedia.org/wiki/Wicked_problem}{Wicked problem}. This is why coming up with a holistic theory of bureaucracy is difficult. 

\href{https://en.wikipedia.org/wiki/Hawthorne_effect}{Hawthorne effect}

As soon as a claim is made, then a group can respond to that claim by behaving in an opposing manner. 

\subsection{Bureaucracy as a Machine}

\textbf{Bureaucracy is a machine that has throughput and latency and dependencies and mechanisms. Bureaucrats are cogs in that machine}.\\
\textit{Why this feels true}: Complex large scale interlocking processes.
\textit{What this is missing}: Perceiving yourself as a cog in the wheel means a loss of agency


\subsection{Bureaucracy as an Economic model}

\subsubsection{A collection of Rational actors}
% https://graphthinking.blogspot.com/2019/05/cooperation-and-competition.html
Individuals cooperate to promote their own self-interests.
Individuals compete to promote their own self-interests.
Individuals enforce social consensus to promote their own self-interests.
Individuals cheat/lie/break social norms to promote their own self-interests.

\subsubsection{Firms}
% https://graphthinking.blogspot.com/2017/09/market-friction-and-bureaucratic.html
Firms exist in a market because negotiating contracts and prices for every interaction is burdensome. 
% https://www.kellogg.northwestern.edu/faculty/hubbard/htm/research/ec174/lectures/3coase.htm

Doesn't address small vs large companies, and doesn't distinguish between profit-oriented and non-profit and government. 


\subsection{Bureaucracy as Emergent phenomenon}
Bureaucracy as a set of many bilateral interactions may not need to invoke emergence. However, there's a universality that hints at emergence. 

Above the threshold for emergence, there is scale-free behavior. The same patterns are observable at large organizations and extremely large organizations.

All those choices faced by the individual are not independent choices with respect to other bureaucrats in their environment. There is a flocking behavior of my choices are informed by the choices of those around me. Not necessarily in space.

Everyone is playing by different rules and has different objectives and everything is dynamic (both individuals and the conditions). 

Bureaucracy as a macroscopic phenomenon is emergent at sufficient scale. The scale is important because there is no longer dependence on individual relationships (beyond \href{https://en.wikipedia.org/wiki/Dunbar\%27s_number}{Dunbar's number}). There are people in the organization that you don't know and for which there is no common accountability. An organization subdivided into team recursively until there is local person-to-person accountability.  

The local rules bureaucrats employ to enable distributed decisions using distributed knowledge is meetings, processes, and communications. 

The relevance of making a claim that something is emergent is that there is behavior occurring at the macroscopic scale, and Knowing that individual motives and actions of every player at the microscopic level is not relevant.

% https://www.preposterousuniverse.com/podcast/2021/10/11/168-anil-seth-on-emergence-information-and-consciousness/
What does ``emergent'' mean? Nominal emergence example: a circle is emergent from a collection of points. Weak emergence is measurable using \href{https://en.wikipedia.org/wiki/Granger_causality}{Granger causality} or, equivalently\footnote{https://arxiv.org/abs/0910.4514}, \href{https://en.wikipedia.org/wiki/Transfer_entropy}{transfer entropy} (information theory). 


A colloquial interpretation of emergent behavior from complex phenomona is treating the system as an entity -- personification.
\marginpar{[Tag] Fallacy}
See \cite{2002_Gall}
\textit{What this is missing}: The problem with treating an org as entity is that apparent behavior is counter-intuitive. But when you break it into individual people their motives make more sense. This merely improve the post hoc rationalization. 

\subsection{Bureaucracy in Game Theory}
Bureaucracy does not fit cleanly into game theory categories of cooperative or competitive.

\href{https://en.wikipedia.org/wiki/Coordination_game}{coordination game}

Maybe all the interactions within a bureaucracy are a bunch of small games?

Bureaucracy is self-modifying. 

Bureaucracy is in constant flux due to external conditions, externally imposed constraints, staff turn-over, internal dilemmas, disagreements of individuals. 




\subsection{Bureaucracy as Evolutionary outcome}

Biological, Genetic -- individual level
Biological, Genetic -- Group selection
Memetic

\subsection{Bureaucracy as Product-focused Narrative}
\textit{Example}: \href{https://www.youtube.com/watch?v=OgVKvqTItto}{School House Rock's I'm Just A Bill}\\

\subsection{Bureaucracy as Subject-focused Narrative}
\textit{Why this feels true}: The person experiencing bureaucracy as a subject is confused by ``why isn't this easier?''  \\
\textit{What this is missing}: History of the processes (legacy) and protection against malicious subjects and protection against malicious bureaucrats. 


\subsection{Bureaucracy as Psychological Phenomenon}

\textbf{Bureaucracy as a pure power struggle}. Or playing out the interplay of individual pyschosis. 

Are you doing what's best for you, the group you're in, or everyone?
Altruistic or reciprocal? Retaliation
The answer changes time the time and situation of situation and person to person

Just a mixture of pathologies?

Perception: Organizations are composed of individuals with personalities, and the inefficiency is attributable to a mashing together of distinct individuals with conflicting desires.
While this is true, it isn't complete. \\
\textit{What this is missing}: Other aspects like history of the processes (legacy) and protection against malicious subjects and protection against malicious bureaucrats. Analysis that stops at personalities misses emergent phenomena. Analysis that stops at personalities of individuals typically explains larger scale phenomena using personification of teams and organizations. 