It is possible to operate within a bureaucracy and entirely focus on ``just do the job.'' It is also possible to do your job and understand the role of engaging with other people -- peer bureaucrats, supervisors, and subjects. In this book a third option of accounting for emergent phenomena has been argued for. 

Learning bureaucracy as a skill doesn't mean you can ignore personalities of individuals. Bureaucracy as a skill is in addition to being a good person, being an effective member of a team, being a good project manager, having technical skills, etc. The distinction from those is that a bureaucrat understands the complications and constraints of their environment and then can more effectively operate within those conditions.


Summary of the "how to be more effective" actions.
Accept that the narrow scope of your role (which hopefully leverages your education/training) does not capture all relevant aspects of your job.
\begin{itemize}
    \item Need to be able to participate or facilitate meetings. This means agenda, providing rules on interaction (raising hands), taking notes, and follow-up.
    \item Focused education on written communication (emails, chat, reports) that empathizes with reader, captures relevant context, concise, clearly worded, 
    \item project management: have a vision, make a plan, all while building consensus
    \item negotiation
\end{itemize}

The attitude to adopt is that of realistic optimism. A realistic optimist will occasionally be wrong but make progress, whereas a pessimist will be right self-fulfilling and not make progress.