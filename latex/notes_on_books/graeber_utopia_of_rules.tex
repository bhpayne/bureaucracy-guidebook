\section{``Utopia of Rules'' by Graeber\label{review:graeber_utopia}}

\cite{2015_Graeber}

review: https://muse.jhu.edu/article/749032/pdf

review: https://journals.sagepub.com/doi/abs/10.1177/0170840615590746?journalCode=ossa

% https://www.amazon.com/gp/product/1612195180

Intended audience:

Ben Payne has read this book: no\\
Ben Payne has a copy: yes, electronic\\
Ben Payne's assessment: written by an anthropologist; gives an outsider's view and the experiences of a subject of bureaucrats. Introduction gives a good history of modern bureaucracy.


This book is comprised of an introduction, three chapters, and an Appendix. 

In the introduction Graeber proposes his ``Iron Law of Liberalism'',
\begin{quote}
    any market reform, any government initiative intended to reduce red tape and promote market forces will have the ultimate effect of increasing the total number of regulations, the total amount of paperwork, and the total number of bureaucrats the government employs.
\end{quote}

Graeber claims working-class Americans see government as comprised of politicians and bureaucrats.

The options for decision making are bureaucracy and the market.


First essay starts with an illustration of the maze.