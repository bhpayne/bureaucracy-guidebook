\section{``Bureaucrat's Handbook'' by O'Hearn\label{review:ohearn_handbook}}

\cite{2015_OHearn}

Intended audience: novice bureaucrats starts their public service career

Ben Payne has read this book: yes\\
Ben Payne has a copy: yes, physical\\
Ben Payne's assessment: terrible; avoid this book!


As a government bureaucrat, I wanted to see if this book could be useful for helping new members of my organization. This book is not appropriate for that use, and the book contains bad advice. Some of the views are harmful. 



In a paragraph about communicating with the public on page 51, the author provides a racist observation: "If the person you are talking to is Asian, and they are nodding when they say yes, assume they don't have the slightest idea of what you're saying, but are too polite to tell you."



In the chapter on education, on page 12 the author's advice is, "To be a good bureaucrat you should become computer literate, as opposed to becoming computer stupid. This is true for any career field you choose. Almost all job applications are submitted on the computer. Office E-Mails have replaced typed memorandums." 



This book was published in 2015, so this content is jarring from the perspective of 2022. I suspect part of the cause is that author is old; the author bio mentions a 48 year career, which means he was probably 70 at the time of publication. 



Setting aside the bad advice and harm in this book, the author's view is that a bureaucrat is a government employee. The author advocates working in Human Resources or a budget office since those are stable jobs. The book does contain some valid observations about life as a bureaucrat; however, the valid insights are outweighed by the number and importance of incorrect explanations. 



The book appears to based wholly on the author's experience. No citations to research and no surveys are mentioned. Only two other publications are cited (Webster's dictionary for definitions, and the book "Rules for Radicals").  The advice is written to be generic to modern (American) bureaucrats. 



The book is self-published. There are a few minor errors (missing periods, extra periods) but not so many as to be disruptive to reading the content. 
