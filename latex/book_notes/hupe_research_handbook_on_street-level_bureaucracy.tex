\section{``Research Handbook on Street-Level Bureaucracy'' by Hupe\label{review:hupe_handbook}}
\cite{2019_Hupe}


Intended audience is researchers.

Ben Payne has read this book: no\\
Ben Payne has a copy: no\\
Ben Payne's assessment:

\begin{quote}
Street-level bureaucracy concerns a vital part of the ways in which public policy programs are implemented, particularly through the relationship between public officials and individual citizens. Addressing the state of the art and providing a systematic exploration of the theoretical and methodological issues at stake, this Research Handbook is a crucial contribution to the analysis of public policy from the perspective of the ground floor of government. 

The Research Handbook covers theoretical themes in current research such as institutional theory, social inequality, national culture, discrimination and representation, digitalization, and accountability. Analysing the role of teachers, police officers and other street-level bureaucrats, chapters explore how these individuals implement policies through their daily contact with citizens. Further sections investigate the methodological tools for research, as well as the future challenges facing the area. Peter Hupe concludes with lessons for the study of street-level bureaucracy and a significant research agenda for the topic.\footnote{\href{https://www.ippapublicpolicy.org/book/research-handbook-on-street-level-bureaucracy/19}{https://www.ippapublicpolicy.org/book/research-handbook-on-street-level-bureaucracy/19}}
\end{quote}