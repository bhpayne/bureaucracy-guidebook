\section{``Street-Level Bureaucracy'' by Lipsky\label{review:lipsky_street}}

\cite{1983_Lipsky}

Intended audience:

author has read this book: yes\\
author has a copy: yes, physical\\
author's assessment:


pages 192 to the end.

Roles: customer, bureaucrat, boss of the bureaucrat

\subsection{Summary of claims}
From \cite{2015_Cooper}
\begin{quote}
Street-level bureaucracy (SLB) is a sociological theory that seeks to explain the working practices and beliefs of front-line workers in public services and the ways in which they enact public policy in their routine work. Developed by an American, Michael Lipsky,1,2 it examines the workplace in terms of systematic and practical dilemmas that must be overcome by employees, with a particular focus on public services such as welfare, policing, and education. The theory is based on the notion that public services represent ‘the coal mines of welfare where the “hard, dirty and dangerous work” of the state’ is done.’3 According to Lipsky,1,2 that is because:

\begin{itemize}
\item demand from clients will always outstrip supply due to finite resources (cost, time, or service access). Most clients are unable to obtain similar services elsewhere (such as private alternatives to state organisations). As a result, employees must resort to ‘mass processing’2 of excessive client caseloads.

\item extensive personal discretion is a critical component of the work of many front-line public sector employees, particularly those who undertake private, face-to-face interaction with clients to assess the credibility of cases. Employees must use their personal discretion to become ‘inventive strategists’ by developing ways of working to resolve excessive workload, complex cases, and ambiguous performance targets.4

    \item employees compromise the quality of their work by ‘creaming off’2 cases that are likely to be straightforward or to have a positive outcome. Alternatively, workers may act as an ‘advocate’2 for clients who are perceived as being at the tip of an iceberg of social vulnerability. Because workers are unable to offer all services to every individual they may be forced to ‘deny the basic humanity’2 of other clients. These pragmatic micro choices ultimately become the de facto policy of the organisation, which may contrast starkly with its official stated aims.
\end{itemize}
This theory has implications not just for the individual employee but also the overall system. In particular, Lipsky suggests that the extensive unmet demand from clients means that even substantial expansion of staff and budgets are unlikely to decrease workload pressures. Instead, he predicted that increased capacity would result in ongoing expansion of the same level of service quality at a higher volume.
\end{quote}


\subsection{Summary of remedies}
% https://graphthinking.blogspot.com/2018/08/how-to-decrease-bureacracy.html
In isolation, none of these ideas should be surprising.
\begin{itemize}
    \item automate recurring decision processes. This decreases bureaucracy by removing subjective influence of bureaucrats
\item where automation is infeasible, make customer advocates with end-to-end authority available
This decreases bureaucracy by improving customer's navigation of processes
\item make processes transparent to participants
This decreases bureaucracy by improving customer's understanding of processes
\item make information discoverable (e.g., via search engine) 
This decreases bureaucracy by 
\item make information directly available, rather than mediated by a person
\item after an interaction is completed, summarize the steps and outcome for the participant
\item when a process fails the needs of a participant, investigate the failure and improve the process
\item make the goals and priorities of the organization clear to all
\item define measurable standards of performance, both for individuals and teams
\item train bureaucrats how to engage participants effectively; these interactions determine the culture
\item train bureaucrats by addressing their immediate problems (e.g., through mentorship)
\item make employment desirable to people who have desirable characteristics (e.g., educated candidates)
\item enhance accountability to peers (e.g., peer review of actions and outcomes)
\item ensure that incentives for organizations and individuals encourage improvement rather than maintenance of status quo
\item decision making should be pushed down the hierarchy to the practitioner
\item when decision making requires cross-organization interaction, form a team of practitioners
\item bosses should share workload with their team in order to gain practical exposure to current challenges
\item seek feedback from process participants; then provide status updates on the implementation process
\end{itemize}
