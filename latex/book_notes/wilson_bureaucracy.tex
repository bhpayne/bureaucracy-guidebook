\section{``Bureaucracy'' by Wilson}

\cite{1991_Wilson}

Intended audience: researchers

author has read this book: no\\
author has a copy: yes, physical\\
author's assessment: good, though there are parts I disagree with


Wilson defines operators as the street-level bureaucrats
Page 33

Agencies typically have (ambiguous) goals, which are separated (subjectively) into tasks
Cite page 34

Incentives (rewards and penalties) matter more than attitude
Cite page 51

Agencies operate under constraints set by Congress; businesses have more freedom to respond to clients

What distinguishes business bureaucracy from government bureaucracy the specified on page 115 as feedback loops and self-determination of scope.

Agencies are "production"(ch8), or procedural/craft, or coping
Cite page 245

Procedural agencies have ambiguous outputs; Craft agencies have invisible operations
Cite page 250

Coping or procedural agencies can discuss their activities but cannot verify their achievements
Cite page 252

In chapter 5, agency environments were classified into four categories: majoritarian, entrepreneurial, clientist, and interest group.
Cite page 248
