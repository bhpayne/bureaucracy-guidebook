\subsection{Communication Preferences}
% why this matters: source of friction among bureaucrats
% benefit to reader: understanding why people are different

There are typically multiple communication channels available between bureaucrats in an organization. Rather than focus on specific implementations using specific technologies, consider two variables. The first variable is asynchronous versus synchronous. Examples of asynchronous communication channels include voice mail, email, web forums, web sites, calendars. Synchronous options include phone calls, video calls, in-person, text-based chat like IRC or Slack. The distinction between these can be blurry, like if someone immediately replies to your (asynchronous) email.  The second variable is a set of categories: voice (e.g., phone), text (e.g., website, web forums, calendars), video (e.g., calls), and in-person. 


The relevance to a bureaucrat of identifying these channels is to know which channel is preferred for what purpose. To be an effective bureaucrat there are reciprocal actions. First, you should \marginpar[list your own preferences so that people know how best to engage with you]{[Tag] Advice}. Second, you should \marginpar[learn and then leverage the preferences of individuals you collaborate with]{[Tag] Advice}.

As an example of knowing your own behaviors, do you keep your calendar up-to-date? Or is your calendar irrelevant? Is that preference stated explicitly to the people who might reference your calendar? Is your calendar visible to your coworkers?

Text chat is useful for asynchronous interrupts. Text chats are better than stopping by in person or calling on the phone since in-person drive-bys and phone calls interrupt whatever I'm thinking about or discussing. The purpose of the text chat is either a reminder or seeking a convenient time to talk. 

Email is useful for notifications or questions or setting up logistics. If I don't respond to a question or request for action within 2 days, please send a reminder. 

Video calls are my preferred method for group meetings. Group meetings via video call scheduled in advance on my calendar are best. Impromptu video calls are acceptable but as interruptive as a phone call. 

In-person discussions are my preferred method for one-on-one discussions. Stopping by my desk  without an appointment interrupts whatever I'm thinking about. If a calendar invitation is inconvenient, a text chat confirming my availability prior to the in-person discussion is helpful. 

Phone is useful when the caller is unable to be present in-person. Phone, being voice-only, is inferior to video calls. Unscheduled phone calls interrupt whatever I'm thinking about. I've intentionally disabled voicemail. 

Status updates in electronic issue trackers or wikis or forums are best (rather than email-only). Notifying in text chat or email that a wiki page or issue is updated is helpful but not mandatory. 

Given all those caveats, I prefer communication in any form at any time over surprises. 

If you're not sure whether you should communicate, default to communicating. If you're not sure whether communication would interrupt something, check my calendar and then communicate. If it's urgent or blocking progress, ignore my calendar. 

\marginpar[Know when to switch communication channels.]{[Tag] Advice} When to start in one channel and then escalate. Text to phone to in person 
