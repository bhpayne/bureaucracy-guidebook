\section{Bad Presentation}

Occasionally I attend a bad presentation. The slides may appear slick but the content is poorly thought out. Or the presenter does not understand the topic well. Or the presenter has good content but does not convey it well. Or the presenter is wrong about the topic. Regardless of the cause, I assume the presenter is making their best effort. 

I could remain silent, complain, criticize, ask leading questions, or offer constructive feedback. My silence may result in other attendees and the presenter leaving with incomplete or wrong information. If I speak up I'll prolong the meeting or limit the presenter's time to convey their material. 

My assessment of the presentation may be wrong. I may lack relevant information. A reliable technique for interjection is to assume a state of confusion instead of confidently asserting the presenter is wrong. 
If I believe the presenter is wrong, asking about the source of the information is a good entry point.

When the information is correct but presented poorly (or above the level I understand), I ask for clarification. 