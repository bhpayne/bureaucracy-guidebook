\subsection{Alternative views of bureaucracy within a bureaucratic organization\label{sec:alternative_views_from_within}}

Bureaucracy as I have defined it in \S\ref{sec:define_bureaucracy} is not the only way that bureaucrats perceive their environment. 

The perspectives below are archetypical; an individual's perspective might be a mixture of these views.

\ \\

\textbf{As a bureaucrat, what matters is what I can accomplish with my skills and the resources I have access to.} \\
\textit{Assessment}: This person is task oriented. Results are what matters. The intricacies of bureaucracy are a distraction to getting the work done. The emotional reward for this person is accomplishment of the task. This person is likely to say to their manager, "Tell me what I need to do to be successful" rather than identify collaborations.

\ \\

\textbf{As a bureaucrat, what matters is how I feel.} \\
\textit{Assessment}: Your feelings are real. They have consequence, in that your emotions impact motivation and enthusiasm. However, a feelings-centric perspective may not be productive for you or your team or the organization. Balancing those competing needs is challenging.

\ \\ 

\textbf{What matters is how others feel.}\\
Depending on the emotional state of those around you is unhealthy and can be unproductive. Working for the happiness or satisfaction of other people is risky -- they may not know what's best, or they may not have your interests in mind.

\ \\

\textbf{What matters is my immediate coworkers.}\\
In this scenario everything else is personified or ignored. This can be positive (I collaborate with those around me) or negative (I am in competition with those around me). \\
\textit{Assessment}: Your do relationships matter. However, they are not all that matters. Being able to explain what is happening outside the immediately observable realm is what is missing from this view. 

