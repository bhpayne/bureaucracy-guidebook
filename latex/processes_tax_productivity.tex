% https://graphthinking.blogspot.com/2021/04/laffer-curve-and-minimum-viable.html

\section{Processes as a tax on Productivity}

The Laffer curve is a claim in economics that there is a relation between government tax rates and the revenue from taxes collected. The relation, based on Rolle's theorem, says that between a tax rate of 0% and 100%, there must be some amount of tax that corresponds to the maximum of revenue. 

While the mathematical statement may be provable, the use in economics seems hand-wavy. In this post, I'll extend that hand-waviness to a different domain: bureaucratic processes in organizations. The relation to the Laffer curve is that bureaucratic processes a tax on productivity. 